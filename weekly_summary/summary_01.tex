\documentclass[11pt]{scrreprt}
\usepackage[utf8]{inputenc}

%change margins here
\usepackage{geometry}

%% This changes default fonts for both text and math mode to use Herman Zapfs
%% excellent Palatino font.  Do not change this.
\usepackage[dvipsnames]{xcolor}
\usepackage[T1]{fontenc}
\usepackage[sfdefault]{FiraSans}

%mathematical typesetting [do not remove]
\usepackage{amssymb, amsmath, amsfonts, amsthm}

%latex own graphics handling
\usepackage{graphicx}

%language package
\usepackage[english]{babel}

%package for header [page layout]
\usepackage{fancyhdr}

%package for hyperlinks (load last for safety)
\usepackage[linkcolor=black,colorlinks=true,citecolor=black,filecolor=black]{hyperref}

\usepackage{math_styling/mathmacros}

\usepackage{enumitem}
\usepackage{bibentry}
\usepackage[style=nature,
    maxnames=2,
    backend=biber,
    safeinputenc,
    isbn=false,
    doi=false,
    maxcitenames=2,
    date=iso8601,
    url=false,
    eprint=false,
    backend=bibtex]{biblatex}

%%%%%%%%%%%%%%%%%%%%%%%%%%%%%%%%%%%%%%%%%%%%%

%Headers and Footers
\pagestyle{fancy}
\fancyhf{}
\fancyhead[L]{FS 2024 \\Master Thesis}
\fancyhead[R]{Nils Jensen \\ 18-943-514}
\fancyfoot[R]{\thepage}
\fancyfoot[L]{nils.jensen@inf.ethz.ch}

%Make line for headers and footers visible
\renewcommand{\headrulewidth}{0.4pt}
\renewcommand{\footrulewidth}{0.4pt}

\addbibresource{bibliography.bib} % Bibliography file

\renewcommand\labelitemi{$\triangleright$}

%%%%%%%%%%%%%%%%%%%%%%%%%%%%%%%%%%%%%%%%%%%%%

\begin{document}
\begin{center}
    {\scshape{\Huge {Summary 1 - \today}}} \\ \vspace{5pt}
\end{center}

\section*{Past Week}

\begin{itemize}
    \item Read the first three chapters of:
          \textit{\citetitle{hollender_structural_2021} from
              \citeauthor{hollender_structural_2021}}
          This gave a good introduction into the definition of the different
          complexity classes, in particular \TFNP, \PPAD, \PLS\ and \EOPL.
    \item Also read the relevant parts of:
          \textit{\citetitle{etessami_tarskis_2020} from
              \citeauthor{etessami_tarskis_2020}}
          This gave an introduction of the \Tarski\ problem and in particular
          the proofs of the membership in \PLS\ and the idea of the membership
          in $\P^{\PPAD} = \PPAD$. In particular I played around with the
          reduction onto the \PLS\ class, and tried the naive approach of using
          this
          reduction to reduce the problem to the \EOPL\ class.
          $\P^{\PPAD}$.
    \item Skimmed over the paper \textit{\citetitle{buss_propositional_2012}
              from
              \citeauthor{buss_propositional_2012}}, seems rather difficult but
          the proof of $\P^{\PPAD} = \PPAD$ might be general enough to be used
          on the
          \EOPL\ class.
\end{itemize}

\section*{Ideas for the next week}

\begin{itemize}
    \item Read \textit{\citetitle{buss_propositional_2012} from
              \citeauthor{buss_propositional_2012}} in more detail, and in
          particular try to understand the proof of $\P^{\PPAD} = \PPAD$, and
          see if it
          can be applied to the \EOPL\ class, i.e. does it hold that $P^{\EOPL}
              = \EOPL$?
    \item Try to find a reduction from \Tarski\ to $\P^{\EOPL} = \EOPL$, which
          should be easier.
\end{itemize}

\section*{Administrative Points}

\begin{itemize}
    \item What are the expectations from your side?
    \item How often should we meet? When will Prof.\ Gärtner be
          available?
    \item Will there be a final presentation?
    \item How spontaneously are you available for questions?
    \item Problems with my Legi.
\end{itemize}
\end{document}
