% Choose the language
\usepackage[english]{babel} % Load characters and hyphenation
\usepackage[english=british]{csquotes}	% English quotes

% Load packages for testing
\usepackage{blindtext}
%\usepackage{showframe} % Uncomment to show boxes around the text area, margin, header and footer
%\usepackage{showlabels} % Uncomment to output the content of \label commands to the document where they are used

% Load the bibliography package
\usepackage{styling/kaobiblio}
\addbibresource{bibliography.bib} % Bibliography file

% Load mathematical packages for theorems and related environments
\usepackage{math_styling/theorems}

% Load the package for hyperreferences
\usepackage{styling/kaorefs}

\graphicspath{{images/}{./}} % Paths where images are looked for

\makeindex[columns=3, title=Alphabetical Index, intoc]
% Make LaTeX produce the files required to compile the index

\newcommand*{\newpartpage}[1]{
	\pagelayout{wide} % No margins
	\addpart{#1}
	\pagelayout{margin} % Restore margins
}

%Extra Packages
\usepackage[notransparent]{svg} %Allows rendering of SVG files
\usepackage{afterpage} %Allows for the creation of empty pages
\usepackage{fontspec} %Allows for the use of custom fonts
\usepackage{math_styling/mathmacros} %Allows for the use of custom math fonts
\usepackage{hyperref}
\usepackage{graphicx}
\usepackage{import}
\usepackage{xifthen}
\usepackage{pdfpages}
\usepackage{textcomp} %Textrightarrow
\usepackage{marvosym} %Bold arrows
\usepackage{newunicodechar} %Allows us to add missing unicode characters
\usepackage{unicode-math} %Allows for the use of unicode math fonts
\usepackage{calc} %Allows for easy length calculations in latex

%Setup for Title Page
\newlength{\drop}
\newcommand*{\titleP}{
    \begingroup
    \drop=0.12\textheight
    \vspace*{\drop}
    \hspace*{0.3\textwidth}
    {\Huge The Complexity}\\[\baselineskip]
    \hspace*{0.3\textwidth}
    {\Huge of Finding Tarski}\\[\baselineskip]
    \hspace*{0.3\textwidth}
    {\Huge Fixed Points}\par
    \vspace*{0.5\drop}
    \hspace*{0.3\textwidth}
    {\Large Master Thesis}\par
    \hspace*{0.3\textwidth}
    {\large \today}\par
    \vspace*{2.5\drop}
    {\large By Nils Jensen}
    \vfill
    {\scshape Advised by}\par
    {\scshape Prof.\ Dr.\ Bernd Gärtner}\par
    {\scshape Sebastian Hasselbacher}\par
    \vspace*{0.5\drop}
    \endgroup
    %Add ETH Logo to the bottom right corner with margin
    \vspace{\baselineskip}
    {\raggedleft{}
        \begin{picture}(0,0)

            \put(-0.3\textwidth,0){\includesvg[width=0.3\textwidth]{images/ethz_logo.svg}}
        \end{picture}
        \par}
}

\renewcommand*{\maketitle}{\titleP{} \afterpage{\blankpage}} %Use custom title page

%Set the font to Fira Math for math symbols
\setmainfont[
	Ligatures=TeX,
	% UprightFont = ,
	ItalicFont = Fira Sans Light Italic,
	% SmallCapsFont = ,
	BoldFont = Fira Sans,
	BoldItalicFont = Fira Sans Italic
]{Fira Sans Light}
%\setsansfont[Ligatures=TeX]{Fira Sans}
%Fix missing characters
\newunicodechar{∩}{\ensuremath{\cap}}

\AtBeginDocument{%
	\DeclareFontShape{TU}{FiraSansLight(0)}{m}{scsl}{<->ssub*lmr/m/sc}{}% %Fira sans does not have slanted small caps, so we use small caps instead
}

%Use Fira Math for Equations
% \usepackage{firamath-otf}
% \setmathfont{TeX Gyre DejaVu Math}[range={\vdots,\ddots}] %Fallback for \vdots and \ddots
% %Fix Mathcal
% \DeclareMathAlphabet{\mathcal}{OMS}{cmbrs}{m}{n}

\setmathfont{Latin Modern Math}

%Fix setminus
\AtBeginDocument{%
	\RenewDocumentCommand{\setminus}{}{\ensuremath{\mathbin{\scalebox{0.75}{\backslash}}}}
}

%Use STIX Two Math for missing math symbols
\setmathfont[range={\subsetneqq}]{STIX Two Math}

\newcommand{\define}[1]{\textit{#1}\index{#1}}

%Easy import of inkscape figures
\newcommand{\incfig}[1]{%
	\def\svgwidth{\textwidth}
	\import{./images/}{#1.pdf_tex}
}

\newcommand{\incfigwide}[1]{%
	\def\svgwidth{\textwidth + \marginparsep + \marginparwidth}
	\import{./images/}{#1.pdf_tex}
}

\newcommand{\marginincfig}[1]{
	\def\svgwidth{\marginparwidth}
	\import{./images/}{#1.pdf_tex}
}

%New command for principles
\newcommand{\principle}[1]{\begin{center}
		``\textit{#1}''
	\end{center}}

%Special marginnote for inside tcolorboxes
\newlength{\marginparseptemp}
\newlength{\characterlength}
\settowidth{\characterlength}{a}
\newcommand{\boxmarginnote}[1]{
	\leavevmode
	%Check if we are in twosided mode
	\ifthenelse{\boolean{twosided}}{
		%We are in two sided mode
		\Ifthispageodd{
			\setlength{\marginparseptemp}{\marginparsep-\theoremboxmargin}
			\marginparsep=\marginparseptemp
			\marginnote[-\baselineskip]{#1}
			\setlength{\marginparseptemp}{\marginparsep+\theoremboxmargin}
			\marginparsep=\marginparseptemp
			\hspace{-\marginparseptemp}
		}{
			\setlength{\marginparseptemp}{\marginparsep+\theoremboxmargin}
			\marginparsep=\marginparseptemp
			\marginnote[-\baselineskip]{#1}
			\setlength{\marginparseptemp}{\marginparsep-\theoremboxmargin}
			\marginparsep=\marginparseptemp
			\hspace{-\marginparseptemp}
		}
	}{
		\setlength{\marginparseptemp}{\marginparsep-\theoremboxmargin}
		\marginparsep=\marginparseptemp
		\marginnote[-\baselineskip]{#1}
		\setlength{\marginparseptemp}{\marginparsep+\theoremboxmargin}
		\marginparsep=\marginparseptemp
		\hspace{-\marginparseptemp}
	}\hspace{\characterlength}\ignorespaces
}

%Settings for the justification of the text
\tolerance=1000
\hyphenpenalty=900

%Setting for page breaks (i.e. no orphans or widows)
\widowpenalty=10000
\clubpenalty=10000

%Remove space before itemize
\setlist{nosep}

%Enumerate settings
\RequirePackage{enumitem}
\setlist[enumerate,1]{leftmargin=7mm, label={(\arabic*)}, font={\bfseries}}% Make left margin smaller
\setlist[enumerate,2]{leftmargin=7mm, label={(\roman*)}, font={\bfseries}}% Make left margin smaller
\setlist[itemize, 1]{leftmargin=7mm}% Make left margin smaller

%Special item tag for problems
\newcommand{\itemspecial}[1]{\item[\textbf{(#1)}]}

%Settings for the algorithm2e package
%Custom comments
\newcommand\mycommfont[1]{\footnotesize\ttfamily\textcolor{blue}{#1}}
\SetCommentSty{mycommfont}

%Spacing for the vertical lines in algorithms
\SetVlineSkip{1mm}

%----------------------------------------------------------------------------------------
%	BOOK INFORMATION
%----------------------------------------------------------------------------------------
\title{The Complexity of Finding Tarski Fixed Points}
\author{Nils Jensen}
\date{\today}

\makeatletter
\def\tcb@split@force@last{%
	\tcb@split@setstate@last%
	\ifdim\tcb@h@total>\tcb@h@page\relax%
		\gdef\tcb@after@lastbox{\clearpage}%
		\tcbdimto\kvtcb@bbbottom{\kvtcb@bbbottom+\tcb@h@page-\tcb@h@total}%
	\fi%
}
\makeatother

%Supress the "marginpar on page X moved." warnings
\usepackage{silence}
\WarningFilter{latex}{Marginpar on page}