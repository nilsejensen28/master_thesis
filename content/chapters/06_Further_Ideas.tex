\setchapterpreamble[u]{\margintoc}
\chapter{Furher ideas on reducing \Tarski\ to \EOPL}\label{ch:eopl_reduction}

Until now we have reduced \Tarski\ to \EndOfLine\ using:
\begin{itemize}
	\item A turing reduction from \Tarski\ to \Tarskistar,
	\item A one-to-many reduction from \Tarskistar\ to \Sperner,
	\item A one-to-many reduction from \Sperner\ to \EndOfLine.
\end{itemize}
In the previous chapter we discussed some limitations as to the structure of the \Sperner\ instance we obtain when reducing monotone \Tarskistar\ to \Sperner. We also showed that these limitations were \emph{not} sufficient as to not allow the \EndOfLine\ instance to have cycles. In this chapter we will discuss ideas on how to overcome these limitations and show that we can reduce \Tarski\ to \EndOfPotentialLine\@. We will ultimately not be able to show that \Tarski\ lies in \EOPL, but hope that the ideas we discuss give some intuition as to the difficulties of the task, and that inuition for the problem can be gained from the ideas we discuss.

%%%%%%%%%%%%%%%%%%%%%%%%%%%%%%%%%%%%%%%%%%%%%%%%%%%%%%%%%%%%%%%%%%%%%%%%%%%%%%%%%%%%%%%%%%%%%%%%%
%Vertex set of an EOPL instance
%%%%%%%%%%%%%%%%%%%%%%%%%%%%%%%%%%%%%%%%%%%%%%%%%%%%%%%%%%%%%%%%%%%%%%%%%%%%%%%%%%%%%%%%%%%%%%%%%
\section{Vertex set of an \EOPL\ instance}[Vertex set of an \EOPL\ instance]

In this section we question the question of what vertex set of a potential \EOPL\ instance obtained by reducing \Tarskistar\ to \EndOfPotentialLine\ would have. Before we do this, we should discuss why this question even arises. Göös et.\ al.\ showed that $\EOPL = \PLS \cap \PPAD$, by constructing a product space of the vertices of both given instances, and defining an \EndOfPotentialLine\ instance on this cartesian product of vertices~\sidecite{goos_further_2022}. We have an \Localopt\ instance given by the states of \cref{alg:iterative_tarski_solver}, whose vertices are basicly the nodes of lattice. We also have an \EndOfLine\ instance given by the vertices of the \Sperner\ instance we obtain when reducing \Tarskistar\ to \Sperner, as discussed in \cref{ch:ppad_reduction}. Here the vertex set, are all the simplices of a simplicial decomposition of the lattice. The question arises, what is the vertex set of the \EndOfPotentialLine\ instance we obtain when reducing \Tarskistar\ to \EndOfLine\@?

The main underlying question is: do we need a vertex set of the size of the cartesian product of the two previous vertex sets, or can we get away with a smaller vertex set? If we cannot use a smaller vertex set, this would make modifying the \EndOfLine\ instance we obtain when reducing \Tarskistar\ to \EndOfLine\ to an \EndOfPotentialLine\ instance much more difficult, as discussed in the next section.

%%%%%%%%%%%%%%%%%%%%%%%%%%%%%%%%%%%%%%%%%%%%%%%%%%%%%%%%%%%%%%%%%%%%%%%%%%%%%%%%%%%%%%%%%%%%%%%%%
%Modifying the EndOfLine instance
%%%%%%%%%%%%%%%%%%%%%%%%%%%%%%%%%%%%%%%%%%%%%%%%%%%%%%%%%%%%%%%%%%%%%%%%%%%%%%%%%%%%%%%%%%%%%%%%%
\section{Modifying the \EndOfLine\ instance}[Modifying the \EndOfLine\ instance]

In the previous chapter we discussed that the \EndOfLine\ instance we obtain when reducing monotone \Tarskistar-instance, has structural limitation induced by the monoticity of the function $f$. The author of this thesis hoped to be able to show, that these structural limitations were strong enough to prevent cycles from existing. This turned out not to be the case as discussed in \cref{sec:existence_of_cycles}. Still the question arises if we can still add a potential to the \EndOfLine\ instance we obtain when reducing \Tarskistar\ to \EndOfLine\@. Of course we would need to be able to handle violations. In particular we would need to be able to use violations of the potential to make progress towards finding a solution. We will discuss an idea on how to handle this in the next section.

Another big challenge when attempting to define a potential on the \EndOfLine\ instance we obtain when reducing \Tarskistar\ to \EndOfLine\ is that the vertex set of the \EndOfLine\ instance is a set of simplices. Defining a potential not on a set of cells, which could be a very similar potential to the one we defined on the lattice with the \Localopt\ instance, but on a set of simplices, is a much more difficult task. We would like the violations of the potential to happen at boundaries of cells, as these are much easier to work with, and have the faces we obtain have nice properties as discussed in the next section. Hence we want to have potential which allows freedom within a cell. This is a challenge would need to be adressed.

%%%%%%%%%%%%%%%%%%%%%%%%%%%%%%%%%%%%%%%%%%%%%%%%%%%%%%%%%%%%%%%%%%%%%%%%%%%%%%%%%%%%%%%%%%%%%%%%%
%Forward and backward points
%%%%%%%%%%%%%%%%%%%%%%%%%%%%%%%%%%%%%%%%%%%%%%%%%%%%%%%%%%%%%%%%%%%%%%%%%%%%%%%%%%%%%%%%%%%%%%%%%
\section{Forward and Backward points}[Forward and Backward points]