\setchapterpreamble[u]{\margintoc}
\chapter{Preliminaries}

The aim of this chapter is to introduce the complexity class \TFNP, and some of its subclasses, in particular \PPAD, \PLS\ and \EOPL.
We will also introduce the \Tarski\ problem.

\section{Total search problems}

The study of complexity classes originally works with so-called \define{decision-problems}, which are the question of deciding on the membership in a set ---
also called a \define{language}.
\marginnote{Notable such problems include deciding on whether a boolean formula can be satified or if a $k$-Clique exist in a given graph.}
Now while these problems are interesting, real world questions or problem often ask for an explicit anwser.
For instance while deciding if a function has a global minimum is a decision problem, we are interrested in actually finding this minimum, which is not a
decision problem.
\marginnote{Even though as we will see it can be transformed into one}
\par
This is where so called \define{search problems} come into play:

\begin{definition}[Search Problem]
    A \define{search problem} is given by a relation $R\subset \binstr \times \binstr$.
    For a given \define{instance} $I\in \binstr$ the computational problem, to find a \define{solution} $s \in \binstr$, that satisfies:
    $(I, s) \in R$ or output ``No'' if no such $s$ exists.
\end{definition}
\marginnote{The ``No'' case can be encoded as some special binary string.}
Now of course we can view these search problems as decision problems by looking at the corresponding decision problem given by the language:
\begin{align*}
    \mathcal{L}_R = \{ I \in \binstr |\ \exists s \in \binstr : (I, s) \in R\}
\end{align*}
We can then ask the classical complexity questions about these search problems, i.e.~whether these search problems are in \P? \NP? whether they are \NP-Hard?
One easily observes that search problems are always at least as hard as just deciding whether a solution exist.
This is because solving a search problem also solves the underlying decision problem.
This leads to the natural question: what if we remove the underlying decision problem?
This can be done by garanteeing that ``No'' is never a solution.
We call these problems where every instance admits a solution \define{total search problems}.
\begin{definition}[Total search problems]
    A \define{total search problem} is a search problem given by a relation $R\subset \binstr \times \binstr$, such that for every given instance $I\in
        \binstr$ there is a solution $s \in \binstr$, that satisfies: $(I, s) \in R$.
\end{definition}
The complexity class \TFNP\ as introduced in \sidecite{papadimitriou_complexity_1994} is simply the class of all total search problems that lie in \NP.
\marginnote{This means that \TFNP\ can be seen as an intermediate class between \P\ and \NP.}
Similarly to decision problem we can also define reduction inside \TFNP.
\begin{definition}[Reduction]
    For two problem $R, S \in \TFNP$, we say that $R$ \define{reduces} to $S$ if there exist polynomial time computable functions $f : \binstr \rightarrow
        \binstr$ and $g : \binstr \times \binstr \rightarrow \binstr$ such that for $I, s \in \binstr$: if $(f(I), s) \in S$ then $(I, g(I, s)) \in R$.
    This means that if $s$ is a solution to an instance $f(I)$ in $S$, we can compute $g(I, s)$ a solution to an instance $I$ in $R$
\end{definition}

\section{An excursion into Binary Circuits}[Binary Circuits]

TODO

\section{Subclasses of \TFNP}

TODO

\subsection{Polynomial Local Search (\PLS)}[\PLS]

TODO

\subsection{Polynomial Parity Argument on Directed Graphs (\PPAD)}[\PPAD]

TODO

\subsection{End of Potential Line (\EOPL)}[\EOPL]

TODO

\section{The \Tarski\ Problem}[\Tarski\ Problem]