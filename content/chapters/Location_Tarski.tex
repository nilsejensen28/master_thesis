\setchapterpreamble[u]{\margintoc}
\chapter{Location of \Tarski\ in \TFNP}

In this chapter, we will discuss where the \Tarski\ problem lies in the complexity landscape of \TFNP. We will start by giving an overview of the known results and explain how they can be combined to obtain that \Tarski\ lies in \EOPL. We will then discuss why it is challenging to chain these results together to obtain a streamlined reduction from \Tarski\ to the \EndOfPotentialLine\ problem, which would gives us intuition into why \Tarski\ lies in \EOPL. This will motivate our approach in the following chapters to circumvent these difficulties.

%----------------------------------------------------------------------------------------
% Location of Tarski in TFNP
%----------------------------------------------------------------------------------------
\section{Summary of known results}[Known results]

We start by summarizing where \Tarski\ lies inside of \TFNP. It has been shown in~\cite{etessami_tarskis_2020} that \Tarski\ lies in \PLS\ as we discussed when presenting \cref{alg:iterative_tarski_solver}. The idea is that the states of this algorithm can be seen as vertices in a directed acyclic graph, and the algorithm can be seen as finding a sink in this graph.

The same paper showed that \Tarski\ lies $\P^{\PPAD}$. We will discuss how this proof works and the reduction of \Tarski\ to $\P^{\PPAD}$ in \cref{ch:ppad_reduction}. On a high level, the idea is to show that one can use the \PPAD-complete problem \Brouwer\ to reduce the subsets of lattice points we need to consider in the \Tarski\ problem by at least half.

Previous work~\sidecite{buss_propositional_2012} showed that many-to-one reductions and Turing-reduction onto \PPAD\ are equivalent. In particular, this means that $\P^{\PPAD} = \PPAD$, and that \Tarski\ lies in \PPAD{}. This proof does not construct a reduction between two complexity classes in the language of complexity theory and of circuit complexity, but rather shows that such a reduction exists using a proof-complexity argument. Trying to unify these two approach and translate the constructions has been an active research direction in the past years, CITE TODO, but it is still not well understood. This means that we do not have a nice reduction between circuits we can leverage.

Now that we have established that \Tarski\ lies inside $\PLS \medcap \PPAD$, we want to discuss the structure of $\PLS \medcap \PPAD$ and describe recent advances in the study of this class. There have been two surprising advances in the study of $\PLS \medcap \PPAD$ in the last few years. The first is that $\CLS = \PLS \medcap \PPAD$ \sidecite{fearnley_complexity_2023}. \CLS\ (Continuous Local Search) was first introduced by Daskalakis and Papadimitriou in \sidecite{daskalakis_continuous_2011} and can be informally thought of as the class of all problems that can be solved by finding the local optimum of a potential in a discrete space equipped with an adjacency relation. The motivation for this class are gradient descent algorithms, which work by deciding locally in which direction to move in order to make progress. It can be shown that finding a Karush–Kuhn–Tucker (KKT) point is a \PLS-complete problem~\cite{daskalakis_continuous_2011}. Which means that \CLS\ intuitively contains the problems that can be solved by gradient descent.

A further notable collapse is the result $\PLS \medcap \PPAD = \EOPL$, which was only recently shown in \sidecite{goos_further_2022}. This implies a full string of equalities:
\begin{align*}
	\EOPL = \CLS = \PLS \medcap \PPAD
\end{align*}
This, intuitively means that solving a problem by using gradient descent, can be reduced to walking along a line with increasing potential. This collapse also implies that \Tarski\ lies in \EOPL.

\begin{figure*}
	\centering
	\incfig{TFNP_Structure}
	\caption[\TFNP-landscape]{Structure of the \TFNP-landscape. Inclusions $\subset$ are indicate by arrows $\rightarrow$.}
	\label{fig:tfnp_structure}
\end{figure*}


TODO: EXPLAIN HOW THE PROOF WORKS

\section{Difficulty of chaining these results together}[Difficulty of chaining results]

This thesis aims to understand why \Tarski\ lies in \EOPL\ and to construct a reduction from \Tarski\ to the \EndOfPotentialLine\ problem. Of course, the first idea that comes to mind is to chain these different results together and achieve a direct reduction in this way. However, this turns out to be very difficult, and not much can be learnt from this exercise. We discuss why this is the case in the following.

The first significant obstacle toward chaining these results together is the proof for the equivalence of many-to-one reductions and Turing reductions onto \PPAD\ \sidecite{buss_propositional_2012}. This proof is based on a proof-complexity argument and does not yield a reduction. This means that we need to translate our results into the language of proof complexity in order to chain this result with the other ones. We propose a different approach, changing our objective from a many-to-one reduction to a Turing reduction of \Tarski\ to the \EndOfPotentialLine\ problem. The reason this is also a valid approach is that \sidecite{buss_propositional_2012} also shows that many-to-one reductions and Turing reductions onto \EOPL\ are equivalent, i.e. $\P^{\EOPL} = \EOPL$. Changing our objective to constructing a Turing reduction will allow us to circumvent the obstacle of using proof complexity.

This means that we will seek to construct a problem \Tarskistar, which is in $\PPAD  \medcap \PLS$ and has the following two properties:
\begin{itemize}
	\item We have a direct one-to-many reduction from \Tarskistar to a \PPAD-complete problem;
	\item We can solve a \Tarski\ instance by solving at most a polynomial number of \Tarskistar instances.
\end{itemize}
We will discuss constructing such a \Tarskistar\ problem in \cref{ch:ppad_reduction}.

The second major obstacle is the proof that $\PLS \medcap \PPAD = \EOPL$ \sidecite{goos_further_2022}. This proof relies on multiple reductions under the hood, most importantly, the paper shows that $\SOPL = \PLS \medcap \PPADS$ first, which implies that $\PLS \cap \PPAD =  \SOPL \cap \PPAD$ because $\PPAD \subset \PPADS$. It then reduces a $\SOPL \cap \PPAD$ problem to a $\EOPL$-complete problem. This means that we first need to follow the proof to find a reduction from \Tarskistar\ to a $\SOPL$-complete problem and then play the same game again to find a reduction from this problem to the \EndOfPotentialLine\ problem. While this is doable, it gives us little insight into why \Tarski\ lies in \EOPL. This is why we decided to attempt a different approach and show that the \EndOfLine\ instance we obtain when reducing \Tarskistar\ to the \EndOfLine\ problem is almost an \EndOfPotentialLine\ instance as we will discuss in \cref{ch:eopl_reduction}.