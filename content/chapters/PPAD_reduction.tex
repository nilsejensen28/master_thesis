\setchapterpreamble[u]{\margintoc}
\chapter{Reducing \Tarski\ to \PPAD}\label{ch:ppad_reduction}

This chapter explores the membership of \Tarski\ in the complexity class \PPAD\@. We begin by presenting a high-level overview of an established proof of the reduction of this problem to \Brouwer~\cite{etessami_tarskis_2020}. We subsequently introduce a novel problem, \Tarskistar, which facilitates a divide-and-conquer approach to solving \Tarski\ by leveraging the structure of the function $f$. This new formulation allows us to provide an alternative proof of \Tarski's membership in \PPAD\ using \textit{Sperner's Lemma} instead of the traditional \textit{Brouwer's Fixed Point Theorem}. This approach simplifies the proof and sets the stage for further reduction of \Tarskistar\ to \EOPL\ in the subsequent chapter.

%%%%%%%%%%%%%%%%%%%%%%%%%%%%%%%%%%%%%%%%%%%%%%%%%%%%%%%%%%%%%%%%%%%%%%%%%%%%%%%%%%%%%%%%%%%%%%%%%
% Presentation of the known reduction of Tarski to PPAD
%%%%%%%%%%%%%%%%%%%%%%%%%%%%%%%%%%%%%%%%%%%%%%%%%%%%%%%%%%%%%%%%%%%%%%%%%%%%%%%%%%%%%%%%%%%%%%%%%
\section{Presentation of the known reduction of \Tarski\ to \PPAD}[Known reduction to \PPAD]

We want to give a high-level presentation of the proof of \Tarski\ membership in \PPAD\ from~\sidecite{etessami_tarskis_2020}, which will help us motivate the introduction of \Tarskistar\ and the subsequent use of \textit{Sperner's Lemma}. The proof given by Etessami et al.\ relies on \textit{Brouwer's fixed point theorem}, which we introduce below.

\begin{theorem}[Brouwer's fixed point theorem]
	Let $K \subset \R^d$ be a compact, convex set. Then every continuous function $f : K \rightarrow K$ has a fixed point $x^*  \in K$, i.e.\ $f(x^*) = x^*$.
\end{theorem}

The original proof can be found in~\sidecite{brouwer_uber_1911}, and a more straightforward proof relying on \textsc{Sperner's Lemma} can be found in~\sidecite{aigner_proofs_2018}. This theorem gives rise to a total search problem, which we call \Brouwer:

\marginnote{We leave out the technical detail of how this function is given using boolean circuits and how precise the output needs to be, as it is irrelevant for this high-level presentation.}
\problem{Brouwer}{
A continuous function $f : K \rightarrow K$.
}{
A fixed point $x^* \in K$ such that $f(x^*) = x^*$.
}

The problem \Brouwer\ was first introduced and shown to be \PPAD-complete in~\sidecite{papadimitriou_complexity_1994}, meaning that it suffices to reduce \Tarski\ to \Brouwer\ in order to show that \Tarski\ is in \PPAD\@. We will construct a Turing reduction of \Tarski\ to \Brouwer, which suffice as \PPAD\ is closed under Turing reductions~\sidecite{buss_propositional_2012}.

The reduction extends the discrete function $f$ to a function $\Tilde{f} : {[0, 2^n - 1]}^d \rightarrow {[0, 2^n - 1]}^d$, such that $\Tilde{f}$ interpolates the lattice function $f$, is piecewise linear between lattice points, and hence continuous. The authors achieve this by using a simplicial decomposition of each cell of the lattice. Now we have an instance of \Brouwer, and hence, we can find a fixed point $x^*$ of $\Tilde{f}$. Of course, this fixed point does not need to be \define{integral}\marginnote{We call a point \emph{integral} if it belongs to the original lattice.}.
The critical insight is that we can use this fixed point to reduce the search area for an integral fixed point by at least half or find a violation of monotonicity. In particular, either there is a fixed point in both $\{x \in {[2^n]}^d : x \geq x^*\}$ and $\{x \in {[2^n]}^d : x \leq x^*\}$, or there is a violation of monotonicity in the cell $x^*$.
We can repeat this procedure, always halving the search area, which allows us to solve a \Tarski\ instance using at most $\BigO{d \cdot n}$ calls to \Brouwer.

%%%%%%%%%%%%%%%%%%%%%%%%%%%%%%%%%%%%%%%%%%%%%%%%%%%%%%%%%%%%%%%%%%%%%%%%%%%%%%%%%%%%%%%%%%%%%%%%%
% Introducing Tarskistar
%%%%%%%%%%%%%%%%%%%%%%%%%%%%%%%%%%%%%%%%%%%%%%%%%%%%%%%%%%%%%%%%%%%%%%%%%%%%%%%%%%%%%%%%%%%%%%%%%
\section{Introducing \Tarskistar}\label{sec:introducing_tarskistar}

In the previous section, we have seen that \Tarski\ can be reduced to a polynomial number of \Brouwer\ instances. We want to study a single such reduction to give an alternative proof that \Tarski\ is in \PPAD\@. In order to do this, we introduce a new problem, \Tarskistar. This problem can be thought of as a subproblem towards solving \Tarski. A standard strategy to solve \Tarski\ is to use a \emph{divide-and-conquer} strategy, for instance, used in~\sidecite{etessami_tarskis_2020}, and presented previously in \cref{alg:recursive_tarski_solver}, of \cref{sec:tarski_algorithms}. We want to construct a problem that allows us to divide the \Tarski\ problem into two smaller problems, where solving the smaller of the two leads to a solution.

For the sake of generality and in order to achieve more precise proofs in the following, we introduce the problem on the integer lattice $L = N_1 \times \cdots \times N_d$, such that $N_i \leq 2^n$ for all $i \in \{1, \dots, d\}$. We propose the following problem:
\problem{\Tarskistar}{
A boolean circuit $f: L \rightarrow L$.
}{
Either:
\begin{itemize}
	\item Two points $x, y \in L$ such that $\norminf{x-y} \leq 1$, $x \leq f(x)$ and $y \geq f(y)$, or;
	\item A violation of monotonicity: Two points $x, y \in L$ such that $x \leq y$ and $f(x) \not\leq f(y)$.
\end{itemize}
}

We want to show that \Tarskistar\ is, in a sense, a subproblem of \Tarski.
\begin{claim}
	An instance of \Tarski\ can be solved using $\BigO{d\cdot n}$ calls to \Tarskistar\ and up to $\BigO{d}$ additional function evaluations.
\end{claim}
\begin{proof}
	We will show that we can use a single call of \Tarskistar\ to either find a violation of monotonicity, a fixpoint, or an instance of \Tarski\, which has at most half as many points and must contain a solution. Let $x, y$ be the two points which a Turing machine solving \Tarskistar\ on a function $f$ outputs. We proceed by case distinction:
	\begin{case}{1}
		We are done if $f(x) = x$ or $f(y) = y$ because we have found a fixpoint.
	\end{case}
	\begin{case}{2.1}
		If $x < y$ and $f(x) \not\leq f(y)$, we have a violation of monotonicity, which solves the given \Tarski\ instance.
	\end{case}
	\begin{case}{2.2}
		If $x < y$ and $f(x) \leq f(y)$, we claim that we can solve the \Tarski\ instance in $\BigO{\normone{x - y}}$ additional function calls. Notice that we have $\norminf{x-y} \leq 1$. Now notice that because $f(x) > x$ (if not, see case 1), there is at least one dimension $i \in \set{1, \dots, d}$ such that $f(x)[i] > x[i]$. Also notice that in this dimension $i$ if $f(y)[i] < y[i]$, then because $\abs{x[i] - y[i]} \leq \norminf{x[i] - y[i]} \leq 1$, we would have a violation of the monotonicity of $f$ in this dimension. Therefore, we must have $f(y)[i] = y[i]$. The same argument shows that if in any dimension $j$ we have $f(y)[j] < y[j]$, then $f(x)[j] = x[j]$. Therefore, we know that because there must be at least one such dimension $i$ and $j$, we have:
		\begin{align*}
			\norminf{f(x) - f(y)} \leq \norminf{x - y} \leq 1 \; \text{and} \; \normone{f(x) - f(y)} \leq \normone{x - y} - 2
		\end{align*}
		Hence, we can now repeat the same argumentation with $f(x)$ and $f(y)$, and we can do this at most $\BigO{\normone{x - y}}$ times until we find a violation of monotonicity or a fixpoint. Because $\normone{x - y} \leq d$, this will take at most $\BigO{d}$ additional steps.
	\end{case}
	\begin{case}{3}
		If $x \not\leq y$, then we can partition the set of lattice points into two sets $S_x$ and $S_y$, as follows:
		\begin{align*}
			S_x = \set{z \in L : z \geq x} \quad \text{and} \quad S_y = \set{z \in L : z \leq y}.
		\end{align*}
		These two sets are disjoint: if there was a $z \in S_x \cap S_y$, then $x \leq z \leq y$, which would imply $x \leq y$, which is a contradiction. Recall that \cref{rem:progress_points_and_subsinstances} shows that there must be a solution of the original instance in $S_x$ and a solution in $S_y$ because both $x$ and $y$ are \emph{progress points}, and that both $S_x$ and $S_y$ together with $f$ restricted to this domain, form valid \Tarski-instances. Hence, it suffices to solve the smaller of the two instances recursively. In particular, because they are disjoint, one of the instances $S_x$ or $S_y$ contains less than half of the lattice points of $L$, and hence we can solve the instance in $\BigO{\log{2^{dn}}} = \BigO{d \cdot n}$ calls of \Tarskistar.
	\end{case}
\end{proof}

For the sake of completeness we should also note that \Tarskistar\ many-to-one reduces to \Tarski. Indeed, every \Tarskistar\ instance also defines a \Tarski\ instance. Solving this \Tarski-instance immediately yields a solution to the \Tarskistar-instance: if the solution is a violation of monotonicity, then it is also a solution of the \Tarski-instance; if the solution is a fixpoint $x^*$ then $x=y=x^*$ is a solution to the \Tarskistar-instance.

Now that we know that \Tarskistar\ is a good stepping stone towards solving \Tarski, we want to investigate why \Tarskistar\ lies in \PPAD\@.

%%%%%%%%%%%%%%%%%%%%%%%%%%%%%%%%%%%%%%%%%%%%%%%%%%%%%%%%%%%%%%%%%%%%%%%%%%%%%%%%%%%%%%%%%%%%%%%%%
% Sperner's Lemma
%%%%%%%%%%%%%%%%%%%%%%%%%%%%%%%%%%%%%%%%%%%%%%%%%%%%%%%%%%%%%%%%%%%%%%%%%%%%%%%%%%%%%%%%%%%%%%%%%
\section{Sperner's Lemma}

This section will discuss how one can prove that \Tarskistar\ lies in \PPAD\@. Rather than employing \textit{Brouwer's fixed point theorem} --- a cornerstone of continuous topology --- we pivot to its discrete analog, \textit{Sperner's Lemma}, a foundational result in combinatorial topology. This approach is particularly apt for two main reasons:
\begin{itemize}
	\item We are working on a discrete lattice, so it seems more natural to use a discrete tool.
	\item Papadimitriou proved that \Brouwer\ is \PPAD-complete by reducing \Brouwer\ to \Sperner\ \sidecite{papadimitriou_complexity_1994}. Hence, by reducing to \Brouwer, we introduce continuity into the problem, which is unnecessary, as it gets removed again behind the scenes.
\end{itemize}

We aim to apply \textit{Sperner's Lemma} on the integer lattice. Using this tool is not directly possible, as \textit{Sperner's Lemma} is defined on a simplicial decomposition of a simplex. Hence, we will first introduce \textit{Sperner's Lemma} for simplices and then show how it can be adapted to work on an integer lattice. We will then formally introduce the total problem \Sperner.

\subsection{Sperner's Lemma for Simplices}[on Simplices]

Before we introduce the Lemma itself, we want to define the setting of the result. We consider a $d$-dimensional simplex\marginnote{By $d$ dimensional simplex we mean the convex Hull of these $d+1$ points in $\R^d$} with vertices $v_0, v_1, \dots, v_d$. We now consider a \define{simplicial subdivision} of this simplex, meaning that we partition the simplex into smaller simplices. We give an example of such a partition in \cref{fig:sperner_setup_example} in the 3-dimensional case.

\begin{figure}[ht]
	\centering
	\incfig{Sperner_Setup_Example}
	\caption[Setup for \textsc{Sperner's Lemma}]{Setup for \textsc{Sperner's Lemma} in the 3-dimensional case. The large simplex spanned by $v_0, v_1, v_2, v_3$ is subdivided into smaller simplices.}\label{fig:sperner_setup_example}
\end{figure}

Now we introduce a coloring $c$ of the vertices of this subdivision with colors $\{0, 1, \dots, d\}$. We want to enforce that the vertices $v_i$ of the large simplex are colored with color $i$ and that the vertices on a face $\{v_{i_0}, \dots, v_{i_k}\}$ are colored with colors $i_0, \dots, i_k$. We give an example of such a coloring in 2 dimensions in \cref{fig:sperner_lemma_example}.

We now introduce Sperner's Lemma, which was first proven in~\sidecite{sperner_neuer_1928}, and for which a more modern proof can be found in~\sidecite{aigner_proofs_2018}.
\begin{theorem}[Sperner's Lemma]
	Suppose a $d$-dimensional simplex with vertices $v_0, \dots, v_d$ is subdivided into smaller simplices. Now color every vertex with a color $\set{0, \dots, d}$ such that $v_i$ is colored $i$, and the vertices on a face $\set{v_{i_0}, \dots, v_{i_k}}$ are colored with colors from the set $\set{i_0, \dots, i_k}$. Then, there is a subsimplex with vertices of every color from the set $\set{0, 1, \dots, d}$.
\end{theorem}

We give an example of a 2-dimensional simplex, subdivided into smaller simplices, and colored according to \textit{Sperner's Lemma} in \cref{fig:sperner_lemma_example}.

\begin{figure}[ht]
	\centering
	\incfig{Sperners_Lemma_Example}
	\caption[Example of \textsc{Sperners Lemma}]{Example of \textsc{Sperners Lemma} in the two-dimensional case, with three colors: orange (0), purple (1), and blue (2). The subsimplex spanned by $v_0$ and $v_1$ only contains blue and purple vertices, the subsimplex spanned by $v_1$ and $v_2$ contains only purple and blue vertices, and the subsimplex spanned by $v_0$ and $v_2$ contains only orange and blue vertices. \textit{Sperner's Lemma} implies that there must be a subsimplex (colored in green) containing all colors.}\label{fig:sperner_lemma_example}
\end{figure}

\subsection{Sperner's Lemma for an integer lattice}[on Lattices]

In the previous section, we introduced \textit{Sperner's Lemma} for a large outer simplex. We want to construct a \TFNP\ problem, which captures the finding of the subsimplex containing all colors. The challenge here is that encoding the structure of this simplex is not straightforward, and hence, defining circuits that encode the setting of the problem will be difficult. Furthermore, our final goal is to reduce \Tarskistar\ to the problem of finding a subsimplex containing all colors. In this case, our setting is not a large outer simplex but an integer lattice.

We want to apply \textit{Sperner's Lemma} to an integer lattice for these reasons. We proceed as follows: We take the $d$-dimensional lattice $L = [N_1] \times \cdots \times [N_d]$, we subdivide each cell into simplices\sidenote{How we do this is not relevant in this chapter but will be discussed in the next chapter.}. We need a technical lemma, which states that the simplicial complex we obtain can be deformed into a large outer simplex.

\begin{lemma}[Simplicial deformation of integer lattice]\label{lem:simplicial_deformation}
	Let $L = [N_1] \times \cdots \times [N_d]$ be a $d$-dimensional lattice. We subdivide each cell into simplices. Then, there is a deformation of the lattice into a large outer simplex.
\end{lemma}
\begin{proof}
	We start by choosing the vertices of the lattice which will form the outer simplex. Consider the following $d+1$ vertices:
	\begin{align*}
		v_0 & = (0, \dots, 0)         \\
		v_1 & = (N_1, 0, \dots, 0)    \\
		v_2 & = (0, N_2, 0, \dots, 0) \\
		    & \vdots                  \\
		v_d & = (0, \dots, 0, N_d)
	\end{align*}
	Now, we need to discuss how we can deform the lattice into this simplex. The idea is the following: Every point on the boundary of the lattices should be moved to the boundary of the simplex. The points in the interior of the lattice should be moved to the interior of the simplex. We can do this by a linear interpolation between the two points. We now give the construction of the deformation function
	\begin{align*}
		D : [0, N_1] \times \cdots \times [0, N_d] \rightarrow [0, N_1] \times \cdots \times [0, N_d]
	\end{align*}
	Notice that we define the deformation function for the hypercube $H = [0, N_1] \times \cdots \times [0, N_d]$ for simplicity. The deformation of the lattice then immediately follows by restricting the deformation function to the lattice.

	Let $x = (x_1, \dots, x_d)$ be a point on the boundary of the hypercube $H$, i.e.\ there is a $i \in \set{1, \dots, d}$ such that $x_i = 0$ or $x_i = N_i$. Notice that we can write:
	\begin{align*}
		x = \sum_{i=1}^{d} \lambda_i \cdot v_i \quad \text{with} \quad \lambda_i \geq 0
	\end{align*}
	Now, we can define the deformation function for these points on the boundary as follows:
	\begin{align*}
		D(x) = \frac{1}{\norm{\lambda}_1} \cdot \sum_{i=1}^{d} \lambda_i \cdot v_i \quad \text{for $x \neq (0, \dots, 0)$}
	\end{align*}
	and $D((0, \dots, 0)) = (0, \dots, 0)$. We immediately notice that if $x$ is on the boundary of the hypercube, then $D(x)$ is on the boundary of the simplex. We can now define the deformation function for the interior of the hypercube by linear interpolation. Let $y$ be a point in the interior of the hypercube; then we can write: $y = \gamma \cdot x$, for some $x$ on the boundary of the hypercube and $\gamma \in [0, 1]$. We can now define the deformation function for the interior of the hypercube as follows:
	\begin{align*}
		D(y) = \gamma \cdot D(x)
	\end{align*}
	This construction has some fundamental properties. Most importantly, segments (edges of the simplicial decomposition) are mapped to segments, and simplices are mapped to simplices. This means that the simplicial decomposition of the lattice is mapped to a simplicial decomposition of the simplex. It follows that we can apply \textit{Sperner's Lemma} to the simplicial decomposition of the lattice. This concludes the proof.
\end{proof}

We give an example of such a subdivision in the 3-dimensional case in \cref{fig:sperner_lattice_example}. Notice that we can deform the lattice and obtain an equivalent simplex and a simplicial decomposition of this simplex.

\begin{figure}
	\centering
	\incfig{Sperner_Lattice_Example}
	\caption[Example of a simplicial decomposition of a lattice]{Example of the simplicial decomposition of a lattice in the three-dimensional case on the left, and the equivalent simplicial decomposition on the right of a simplex $v_0, v_1, v_2, v_3$.}\label{fig:sperner_lattice_example}
\end{figure}

Assuming that we color all vertices of the lattice with colors $\set{0, \dots, d}$, such that $v_i$ is colored $i$, and every vertex $x$ with $x[i] = 0$, is \textit{not} colored $i$ for $i \in \set{1, \dots, d}$. Then, we can apply \textit{Sperner's Lemma} to this simplicial decomposition of the lattice, and we will find a simplex that contains all colors. Of course, because every subsimplex is included in exactly one cell by construction, there must be a cell that contains all colors. This motivates the definition of the total problem \Sperner, which was introduced in~\sidecite{papadimitriou_complexity_1994}. We introduce the problem for a general lattice $L = N_1 \times \cdots \times N_d$, such that $N_i \leq 2^n$, and define distinguished vertices $v_0, \dots, v_d$ given by\marginnote{$\delta_{i, j}$ is the Kronecker delta given by: \begin{align*}
		\delta_{i, j} = \begin{cases}
			                0 & \text{ if $i \neq j$,} \\
			                1 & \text{ if $i = j$}
		                \end{cases}
	\end{align*}}
\begin{align*}
	v_i[j] & := \delta_{i, j} \dots N_j
\end{align*}
which correspond to the outer vertices of the simplex. The \Sperner-problem is then given by:

\problem{\Sperner}{
A coloring $c : L \rightarrow \set{0, \dots, d}$ of the vertices of $L$, such that:
\begin{itemize}
	\item for every $i \in \set{1, \dots, d}$ the the vertices $\set{x \in L : x[i] = 0}$ are not colored $i$ and
	\item for all $i \in \set{0, \dots, d}$ we have $c(v_i) = i$.
\end{itemize}
}{
Either:
\begin{itemize}
	\item A cell $C$ such that for all $i \in \set{0, \dots, d}$ there is a vertex $x \in C$ such that $c(x) = i$, or;
	\item A violation of the assumptions of the input.
\end{itemize}
}

\subsection{Reducing \Sperner\ to \EndOfLine}[\Sperner\ to \EndOfLine\ reduction]\label{sec:sperner_eol_reduction}

This section will discuss the reduction of \Sperner\ to \EndOfLine. This reduction was first constructed in the three-dimensional case in~\sidecite{papadimitriou_complexity_1994}. The same paper also gives the idea of the generalization to the $d$-dimensional case. In the following section, we will give the complete construction of the reduction. This is important as we will use this reduction to argue as to why \Tarskistar\ is in \EOPL\@.

\begin{theorem}[\Sperner\ is in \PPAD]
	\Sperner\ reduces to \EndOfLine.
\end{theorem}

We start by giving the idea of the construction. We want to find a cell that contains all colors: assume that we start at a cell with all but one color $d$. Then, we move to the neighboring cell through a face containing all colors but $d$. Now, either this cell contains the color $d$, in which case we are finished, or we have a second face containing all colors but $d$. We can repeat this process until we find a cell containing all colors.


Now, there are two problems we need to discuss. First, once again, using a cell leads to some difficulty as there could be more than two faces for a given cell, which could contain all colors but one. We will solve this problem again by considering the lattice's simplicial decomposition. Second, we need to define a designated source. In order to do this, we will expand the simplicial complex slightly. We are now ready for the formal proof; we recommend the reader to follow along with the construction in \cref{fig:sperner_eol_reduction}.

\begin{figure}
	\centering
	\incfig{Sperner_EoL_Reduction}
	\caption[Reduction of \Sperner\ to \EndOfLine]{Reduction of two-dimensional \Sperner\ to \EndOfLine. The circuit $S$ is given by the full arrows and $S$ by the dashed arrows. The solutions are colored green. The designated source is colored blue.}\label{fig:sperner_eol_reduction}
\end{figure}

\begin{proof}
	Formally, we will proceed by induction over the dimensions of the lattice. First, we discuss the base case. We have a lattice $L = [N]$ in the one dimensional case. We color the lattice with colors $\set{0, 1}$. Now for every segment $s = [x, x+1]$, of which there are $N-1$ which we number from left to right of the lattice, we define the circuits $S, P : [N-1] \rightarrow [N-1]$ as follows:
	\begin{align*}
		S(x) & = \begin{cases}
			         x + 1 & \text{if $c(x+1) = 0$} \\
			         x     & \text{else}
		         \end{cases} \quad \text{and} \\
		P(x) & = \begin{cases}
			         x - 1 & \text{if $c(x-1) = 0$} \\
			         x     & \text{else}
		         \end{cases}
	\end{align*}
	Furthermore set $P(0) = \xtilde$ and $S(N-1) = N-1$. By adding a designated source $\xtilde$ outside of the lattice and setting $P(\xtilde) = \xtilde$ and $S(\xtilde) = 0$, we obtain an instance of \EndOfLine.

	Now, we are ready to proceed with the induction. Assume the theorem holds for all dimensions $1, \dots, d-1$. We will prove it for dimension $d$.

	Consider the simplicial subdivision of the colored lattice. A given cell is subdivided into ${d!}$ simplices\marginnote{A more detailed discussion of this is discussed in \cref{sec:freudenthal_simplicial_decomposition}.}. It follows that we have $N = N_1 \cdot \cdots \cdot N_d \cdot {d!}$ simplices, which we can number from $0$ to $N-1$.

	We now can define the circuit $S, P : [N] \rightarrow [N]$. For a given $d$-simplex $x \in [N]$, we consider the color of the vertices. We proceed by case distinction:
	\begin{case}{1}
		If $x$ has a vertex with every color in $\set{0, \dots, d-1}$, but no vertex colored $d$, then $x$ has two faces colored with all colors but $d$. One of these faces is oriented positively, and one is oriented negatively.\boxmarginnote{See \cref{sec:orientation_of_simplex} for a detailed discussion of how we define these orientations.}Now, define $S(x)$ as the simplex obtained by moving from $x$ through the positively oriented face and define $P(x)$ as the simplex obtained by moving to the negatively oriented face.
	\end{case}
	\begin{case}{2}
		If $x$ is a simplex with all colors, look at the face spanned by the vertices colored $\set{0, \dots, d-1}$. If this face is positively oriented, then define $S(x)$ as the simplex obtained by moving through this face and $P(x) = x$. If this face is oriented negatively, then define $P(x)$ as the simplex obtained by moving through this face and $S(x) = x$. Notice that these are the sources/sinks of the circuit and the solutions to the \Sperner\ instances.
	\end{case}
	\begin{case}{3}
		We define $S(x) = P(x) = x$ in all other cases.
	\end{case}
	Notice that we can compute $S$ and $P$ in polynomial time with respect to $d$. We still need to define a distinguished source. Finding a distinguished source can be seen as finding a face colored with all colors $\set{0, \dots, d-1}$ on the face spanned by the vertices $v_0, \dots, v_{d-1}$. We can do this by solving a $d-1$ dimensional \Sperner\ instance on this face. By induction hypothesis, this can be reduced to an \EndOfLine\ instance. We can now define the distinguished source as the source of this \EndOfLine\ instance and slightly modify the circuit $S$ and $P$ to combine this \EndOfLine\ instance with the circuits $S$ and $P$ we have constructed above. This concludes the proof.
\end{proof}

Next, we will use this problem to show that \Tarskistar\ reduces to \Sperner\ and hence lies in \PPAD\@.

%%%%%%%%%%%%%%%%%%%%%%%%%%%%%%%%%%%%%%%%%%%%%%%%%%%%%%%%%%%%%%%%%%%%%%%%%%%%%%%%%%%%%%%%%%%%%%%%%
% Reducing Tarskistar to Sperner
%%%%%%%%%%%%%%%%%%%%%%%%%%%%%%%%%%%%%%%%%%%%%%%%%%%%%%%%%%%%%%%%%%%%%%%%%%%%%%%%%%%%%%%%%%%%%%%%%
\section{Reducing \Tarskistar\ to \Sperner}\label{sec:tarskistar_to_sperner}

For us to be able to use \textsc{Sperner's} Lemma on our \Tarskistar\ instances, we need to define a coloring of the vertices of $L$. We propose the following coloring $c : L \rightarrow \set{0, \dots, d}$:
\marginnote{A vertex colored 0 indicates that the function points \emph{weakly forwards} in all dimensions, a vertex colored $i$ for $i \geq 1$ indicates that the function points \emph{backwards} in at least the $i$-th dimension.}
\begin{align*}
	c(x) =
	\begin{cases}
		0 & \text{if $x \leq f(x)$}         \\
		1 & \text{else if $x[1] > f(x)[1]$} \\
		  & \vdots                          \\
		d & \text{else if $x[d] > f(x)[d]$}
	\end{cases}
\end{align*}

\begin{figure}[ht]
	\centering
	\incfig{Tarski_Coloring_Example}
	\caption[Coloring of a \Tarskistar-instance]{Coloring of a \Tarskistar\ instance on a 2-dimensional lattice. The vertices colored 0 indicate that the function points weakly forward in all dimensions, the vertices colored 1 indicate that the function points backward in the first dimension, and the vertices colored 2 indicate that the function points backward in the second dimension and not in the first.}\label{fig:tarskistar_coloring}
\end{figure}

We give an example of the coloring of a \Tarski-instance in \cref{fig:tarskistar_coloring}. We now need two results. We will distinguish two cases:
\begin{enumerate}
	\item when for all $i \in \set{0, \dots, d}$ we have $c(v_i) = i$;
	\item when there is an $i \in \set{0, \dots, d}$ such that $c(v_i) \neq i$.
\end{enumerate}
We will show that in both cases \Tarskistar\ can be reduced onto \Sperner. We will start by the first case and then discuss why the second case can be reduced onto the first case. Hence, for now assume that the \Tarskistar-instance is such that $c(v_i) = i$ for all $i$. First, we need to show that a cell with all colors always exists, allowing us to show that \Tarskistar\ is a total search problem. Second, we need to show that finding a cell with all colors yields a solution to \Tarskistar\ in polynomial time.

\begin{claim}
	For any \Tarskistar\ instance with vertices colored as above, there is always a cell with all colors.
\end{claim}
\begin{proof}
	This claim follows directly from \textsc{Sperner's} Lemma and the coloring we have defined. There can never be a vertex colored $i$ with $x[i] = 0$ because this would imply that $f(x)[i] < x[i]$, which is a contradiction to the construction of the function. Hence, by dividing each cell of the lattice into simplices, we can apply \textsc{Sperner's} Lemma to show that a cell with all colors always exists. The vertices we use as the vertices of the large simplex are $\set{(0, \dots, 0), (2^n - 1, 0, \dots, 0), \dots, (0, \dots, 2^n - 1)}$.
\end{proof}
\begin{claim}
	Finding a cell with all colors yields a solution to \Tarskistar, in $\BigO{d}$ additional steps.
\end{claim}
\begin{proof}
	Assume we have found a cell, with vertices colored $0, \dots, d$. Let us denote $x_i$ the vertex colored $i$, for $i \in \set{0, \dots, d}$. Notice that all of these vertices are by construction contained in some cell (hypercube of length $1$); let $\mathbf{0}$ be the smallest vertex of this hypercube and $\mathbf{1}$ the largest. In particular, this means that for all $i$, we have:
	\begin{align*}
		\mathbf{0} \leq x_i \leq \mathbf{1} \quad \text{and} \quad f(x_i)[i] < x_i[i] \quad \text{for $i > 0$}
	\end{align*}
	We now proceed by case distinction:
	\begin{case}{1}
		If $x_0$ is a fixed point, then $x = y = x_0$ is a solution to \Tarskistar.
	\end{case}
	\begin{case}{2}
		If $x_0 \neq f(x_0)$ and $x_0 = \mathbf{0}$. Then there is an $i$ such that $f(x_0)[i] > x_0[i]$, which means that $f(x_0[i]) - x_0[i] \geq 1$. At the same time we must have $f(x_i)[i] < x_i[i]$ and $x_0[i] - x_i[i] \leq 0$ because $x_0 = \mathbf{0}$, and hence $x_i[i] - f(x_i)[i] \geq 1$. Now we get:
		\begin{align*}
			f(x_0)[i] - f(x_i)[i] & = \underbrace{f(x_0)[i] - x_0[i]}_{\geq 1} + \underbrace{x_0[i] - x_i[i]}_{\geq 0} + \underbrace{x_i[i] - f(x_i)[i]}_{\geq 1} \\
			f(x_0)[i] - f(x_i)[i] & \geq 2
		\end{align*}
		This implies that $f(x_0) \not \leq f(x_i)$, and hence $x_0, x_i$ are two points witnessing a violation of monotonicity of $f$, which form a solution to \Tarskistar.
	\end{case}
	\begin{case}{3}
		If $x_0 \neq f(x_0)$ and $x_0 \neq \mathbf{0}$. We claim that either $f(\mathbf{0}) \leq \mathbf{0}$, or we have a violation of monotonicity. Assume for the sake of contradiction that there is an $i$ such that $f(\mathbf{0})[i] > \mathbf{0}[i]$. Then we must have $f(x_i)[i] < x_i[i]$ hence we get: $f(\mathbf{0})[i] \not\leq f(x_i)[i]$, which is a violation of monotonicity. This means that either we can return $y = x_0$ and $x = \mathbf{0}$ as a solution to \Tarskistar, or $x_i$ and $\mathbf{0}$ as a violation of monotonicity.
	\end{case}
	This shows we can solve a \Tarskistar\ instance in $\BigO{d}$ additional steps.
\end{proof}

This shows that in the first case we can reduce \Tarskistar\ to \Sperner. Now we need to discuss the second case, if for some $i \in \set{0, \dots, d}$ we have $c(v_i) \neq i$, notice that this means that $c(v_i) = 0$, as any other coloring would mean that $f(v_i)$ is no longer in the lattice. But this means that $v_i$ is a forward-point, and hence that $L_{\geq v_i}$ is a valid $d-1$ dimensional \Tarskistar-instance, by \cref{rem:progress_points_and_subsinstances}. Now we can repeat the same argument for this $d-1$ dimensional \Tarskistar-instance, if we are in the first case we are done, as this can be reduced to \Sperner. If we are in the second case we repeat the argument we just made. Repeating this argument we must obtain an \Tarskistar-instance which corresponds to the first case, or a one-dimensional \Tarskistar-instance. We must now only show that this one-dimensional instance \emph{always} reduces to \Sperner.

\begin{claim}[One-dimensional \Tarskistar\ reduces to \Sperner]\label{claim:tarski_one_dimensional_reduction}
	A one dimensional \Tarskistar-instance always reduces to \Sperner\@.
\end{claim}

\begin{figure}
	\centering
	\incfig{1D_Tarskistar_Sperner_Reduction}
	\caption[Reduction of one-dimensional \Tarskistar\ to \Sperner]{Setting for the proof of \cref{claim:tarski_one_dimensional_reduction}}\label{fig:tarski_one_dimensional_reduction}
\end{figure}

\begin{proof}
	Let $v_0$ and $v_1$ be both ends of the lattice (which is a line). Now color all other vertices as previously discussed, that is:
	\begin{align*}
		c(x) =  \begin{cases}
			        0 & \text{ if $f(x) \geq x$} \\
			        1 & \text{ else}
		        \end{cases}
	\end{align*}
	Notice that because we only have one dimension, and $f$ stays in the lattice we must have $c(v_0) = 0$ and $c(v_1) = 1$. Now we add a second line of the same length, in parallel, and add edges to form a lattice, as shown in \cref{fig:tarski_one_dimensional_reduction}. We color these new vertices with the color $2$. This forms a valid \Sperner\ instance.
\end{proof}

This shows that in both cases \Tarskistar\ can be reduced to \Sperner. Hence, \Tarskistar\ lies in \PPAD, and by using that $\P^{\PPAD} = \PPAD$, we have shown that \Tarski\ lies in \PPAD, without relying on \Brouwer.