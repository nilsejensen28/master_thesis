\setchapterpreamble[u]{\margintoc}
\chapter{Preliminaries}\label{ch:preliminaries}

This chapter aims to establish the complexity framework used throughout this thesis to study the \Tarski\ problem. It formally introduces the concept of total search problems, the complexity class \TFNP, and its subclasses \PLS, \PPAD, and \EOPL\@. In addition, in this chapter, we will describe how we represent sets and functions in this framework and how their complexity is measured. Finally, we give a formal introduction to the \Tarski\ problem, a presentation of some known algorithms for solving it, aswell as promise and uniqueness variants of \Tarski\@.

%%%%%%%%%%%%%%%%%%%%%%%%%%%%%%%%%%%%%%%%%%%%%%%%%%%%%%%%%%%%%%%%%%%%%%%%%%%%%%%%%%%%%%%%%%%%%%%%%
% Total search problems
%%%%%%%%%%%%%%%%%%%%%%%%%%%%%%%%%%%%%%%%%%%%%%%%%%%%%%%%%%%%%%%%%%%%%%%%%%%%%%%%%%%%%%%%%%%%%%%%%
\section{Total search problems}
Subclasses of TFNP
\begin{definition}[Search Problem]
	\boxmarginnote{The ``No'' case can be encoded as some special binary string.}
	A \define{search problem} is given by a relation $R\subset \binstr \times \binstr$. For a given \define{instance} $I\in \binstr$, the computational problem is to find a \define{solution} $s \in \binstr$ that satisfies $(I, s) \in R$, or output ``No'' if no such $s$ exists.
\end{definition}

We can view these search problems as decision problems by looking at the corresponding decision problem given by the language:
\marginnote[10mm]{Here, we have rephrased the valid language as the pair of a problem instance and a valid solution.}
\begin{align*}
	\mathcal{L}_R = \{ I \in \binstr |\ \exists s \in \binstr : (I, s) \in R\}
\end{align*}

The above shows that every search problem can be seen as a decision problem of a broader language. This perspective allows us to ask classical complexity questions about search problems: Are these problems in \P\ or \NP\@? Are they \NP-hard? It is evident that search problems are at least as complex as their decision counterparts, since solving a search problem inherently solves the associated decision question.

Similarly to decision problems, we want to study which problems can be solved efficiently and which cannot. This question leads us to the definition of the complexity class \FNP, which is pendant for search problems, to  \NP\ for decision problems. We introduce \FNP\ formally as in~\sidecite{megiddo_total_1991}.

\begin{definition}[Function \NP\ (\FNP)]
	We say that a relation $R\subset \binstr \times \binstr$ is in \FNP\ if it is:
	\begin{itemize}
		\item \define{polynomial-time recognizable}, i.e.\ there is a Turing machine $M$ which decides in polynomial time if a given pair $(I, s)$ is in $R$; and
		\item  \define{polynomially balanced}, i.e.\ there exists a polynomial $p$ such that for every $(I, s) \in R$ implies $\length{s} \leq p(\length{I})$.
	\end{itemize}
	The class \FNP\ is the set of all relations $R$ that are polynomially balanced and recognizable.
\end{definition}

This means that for a relation $R$ in, \FNP\ we can check the validity of a solution to the associated search problem in polynomial time. Of course, in the general case all instances $I$ do not have a solution $s$, such that $(I, s)$, and deciding if such an $s$ exists is the question which we aim to answer when working with decision problems. This means that finding a solution to a \FNP\ instance can in some sense be seen as also solving the underlying decision problem. Removing this natural counterpart, by guaranteeing that every instance has a solution, is what leads to the definition of a total search problem~\cite{megiddo_total_1991}:

\begin{definition}[Total search problems]
	\boxmarginnote{This means that a solution always exists for any input, i.e.\ ``No'' is never a valid answer.}
	A \define{total search problem} is a search problem given by a relation $R\subset \binstr \times \binstr$, such that for every given instance $I\in
		\binstr$ there is a solution $s \in \binstr$ that satisfies: $(I, s) \in R$.
\end{definition}

The complexity class \TFNP\ is simply the class of all \emph{total} problems in \FNP\@.

\begin{definition}[Total Function \NP\ (\TFNP)]
	The class \TFNP\ is the set of all problems given by total relations which lie in \FNP\@.
\end{definition}

Examples of \TFNP\ problems are:
\begin{itemize}
	\item \textsc{Factoring}, the problem of finding the prime factors of a natural number. Every natural number admits a factorization into prime numbers, which can be checked in polynomial time;
	\item \textsc{Nash}, the problem of finding a Nash equilibrium in a bimatrix game~\sidecite{daskalakis_complexity_2009};
	\item \textsc{Minimize}, the problem of finding the global minimum of a convex function~\sidecite{daskalakis_continuous_2011}.
\end{itemize}
Similarly to \P, \FP\ is the class of all search problems that can be solved in polynomial time. It is not known whether \FP\ is equal to \TFNP, but it is widely believed --- similarly to \P\ and \NP\ --- that they are different.

\subsection{Reductions}

Similarly to decision problems, we can also define reductions inside \TFNP\@.

\begin{definition}[Many-to-one Reduction]
	\boxmarginnote{Many-to-one reductions are also sometimes called \define{Karp reductions} in the litterature.}
	For two problems $R, S \in \TFNP$, we say that $R$ \emph{reduces} (many to one) to $S$ if there exist polynomial time computable functions $f : \binstr \rightarrow \binstr$ and $g : \binstr \times \binstr \rightarrow \binstr$ such that for $I, s \in \binstr$:
	\begin{align*}
		\text{If } (f(I), s) \in S \text{ then } (I, g(I, s)) \in R.
	\end{align*}
	\boxmarginnote{Saying that \emph{one can reduce $R$ onto $S$} can be understood as saying that \emph{if one can solve $S$ efficiently, then one can solve $R$ efficiently}.}
	This means that if $s$ is a solution to the instance $f(I)$ in $S$, we can compute a solution $g(I, s)$ to an instance $I$ in $R$
\end{definition}

Notice that many-to-one reductions map \emph{instances} to \emph{instances}. There is a more general type of reduction, which allows us to use multiple instances of $S$ to reduce $R$ onto $S$, which we introduce next:

\begin{definition}[Turing Reduction]
	For two problems $R, S \in \TFNP$, we say that $R$ \emph{Turing reduces} to $S$ if there exists a polynomial-time oracle Turing machine that solves $R$ given access to an oracle\boxmarginnote{An \define{oracle} is a black-box which solves $S$ in one computational step.} for $S$.
\end{definition}

Note that if $R$ \emph{Turing}-reduces to $S$, then $R$ also \emph{many-to-one}-reduces to $S$. The reverse implication is generally not true, but is known to hold inside specific complexity classes~\sidecite{buss_propositional_2012}, some of which we will study in this thesis. We will present these results in \cref{ch:location_tarski}.

\subsection{Promise Problems}\label{sec:promise_problems}

We have defined total search problems as problems where a solution always exists for \emph{any} input in ${\{0, 1\}}^*$. However, in practice, we often study problems where a solution is guaranteed to exist only for a subset of the inputs. For instance, every convex function has a global minimum, but this existence result relies on the fact that we are given a \emph{convex} function. This leads us to the notion of \define{promise problems} as introduced in~\sidecite{hollender_structural_2021}. Formally, we restrict the instance space to some subset $\mathcal{X} \subset {\{0, 1\}}^*$. We only require our algorithm to solve the problem for instances in $\mathcal{X}$, and it can behave arbitrarily on instances outside of $\mathcal{X}$.

We highlight that formally \TFNP\ does not contain promise problems where $\mathcal{X} \neq {\{0, 1\}}^*$. We still want to study these problems inside the \TFNP-framework. There is a trick for restricting the input space to a subset $\mathcal{X} \subset \set{0, 1}^*$, where the language $\mathcal{X}$ can be decided in polynomial time. For a promise problem $R$ on $X \subset {\{0, 1\}}^*$, which is both polynomial-time recognizable and polynomially bounded, we can define the total search problem $R'$ on ${\{0, 1\}}^*$ by adding a solution $(I, \star)$ to $R$ for all $I \in {\{0, 1\}}^* \setminus R$, where $\star$ is some special binary string. Because it can be decided in polynomial time whether an instance is in $\mathcal{X}$, we can solve $R'$ by checking whether the instance is in $\mathcal{X}$ and then solving $R$, hence obtaining a problem in \TFNP, due to the assumption that $R$ is polynomial-time recognizable and polynomially balanced.

For example, in this thesis, we use syntactic validation when the instances are a function or a boolean circuit to validate that the given input is indeed an encoding of a function or circuit. This verification can be done in polynomial time~\sidecite{greenlaw_chapter_1998}, and a special binary string can be outputted if this verification fails. For example, this is the case for the \Tarski\ problem, where the instances are boolean circuits, and the validity of the instances can be checked in polynomial time. For the sake of simplicity, we will assume this step implicitly when defining \TFNP-Problems, and allow instances which have an input space $\mathcal{X} \subset \set{0, 1}^*$ if it can be validated in polynomial time.

In practice, we often use \define{violations} to enfore synctatic limits. For example, if we are interested in finding the global minima of convex functions, we can construct a total search problem by:
\begin{enumerate}
	\item Checking syntactically that the input defines a function;
	\item Adding a violation of convexity to the solution space\marginnote{A violation of convexity is given by a $x, y \in \mathcal{D}_f$, and $t \in \{0, 1\}$ such that $t f(x) + (1-t)f(y) < f(tx + (1-t)y)$.}. Formally, this is done by changing the relation $R$ to ensure that a solution exists for every instance $I$; this can be thought of as allowing more solutions.
\end{enumerate}

We conclude that we can often construct a \TFNP-problem starting from a promise-problem by checking the validity of the input syntactically and adding violations to the solution space. However, it is essential to note that this is only sometimes the case and that constructing a \TFNP\ problem from a promise problem can be a non-trivial task. Also, there is no unique way of constructing \TFNP-Problems from promise problems, and care has to be taken to introduce the studied problem rigorously.

\subsection{Representation of the problems}[Representing problems]

As we discussed in the previous subsection, great care has to be taken, when describing the inputs of the problems we will be working with. We will mainly work with sets and functions as describing the input of problems. In this subsection, we describe how we denote these objects in this thesis, and how we formally encode them.

\subsubsection{Representation of sets}

In this thesis, we will work with sets of the form $S = \{0, \dots, 2^n - 1\}$, which we will denote by $[2^n]$\marginnote{Notice that we do \emph{not} include $n$ in $[n]$.}. Notice that this set can be identified with the set of binary strings of length $n$. We will denote the set of binary strings of length $n$ by $\bitstr^n$. Formally, the functions and the model we will use to represent the functions will use the underlying binary strings in $\bitstr^n$. We often denote the integer $x \in [2^n]$ interchangeably with its representation as a binary string.

Similarly, when considering the $d$-dimensional case, we can represent the set $L = {[2^n]}^d$, which corresponds to a $d$-dimensional lattice with side length $2^n$, as the set of binary strings of length $n \cdot d$, i.e.\ $\bitstr^{nd}$. Again, for simplicity, while the underlying functions rely on the binary strings, these naturally correspond to a unique point $(x_1, \dots, x_d) \in {[2^n]}^d$. We will use both notations interchangeably.

\subsubsection{Representation of functions}

Now that we have described the sets, we can describe how we represent the functions we will use in our problem instances. We will represent the functions by using so-called boolean circuits. In this section, we will rely on the presentation of boolean circuits described in~\sidecite{greenlaw_chapter_1998} and refer an interested reader to this source for a more detailed description.

On a high level, a boolean circuit is a directed acyclic graph, where the nodes are called \define{gates}, and the edges are called \define{wires}. The sinks of the graphs are the output gates, and the sources are the input gates. We want to start by defining a gate formally.

\begin{definition}[Gate]
	\boxmarginnote{This corresponds to the gate node, having $k$ incoming edges, and one outgoing edge.}
	A gate is a function $g : \bitstr^k \rightarrow \bitstr$, where $k$ is the number of input wires of the gate.
\end{definition}

In this thesis, we will only consider the following types of gates:
\begin{itemize}
	\marginnote{Notice that we only consider gates with at most two inputs, as we can always represent a gate with $k$ inputs as a composition of gates with at most two inputs.}
	\item \textbf{AND-gate}: $g(x_1, x_2) = x_1 \land x_2$,
	\item \textbf{OR-gate}: $g(x_1, x_2) = x_1 \lor x_2$,
	\item \textbf{NOT-gate}: $g(x) = \lnot x$.
\end{itemize}

Now, we can describe a boolean circuit formally as follows:
\begin{definition}[Boolean circuit]
	A boolean circuit $C$ is a labeled finite directed acyclic graph, where each vertex has a \define{type} $\tau$, with
	\begin{align*}
		\tau(v) \in \{\text{INPUT}\} \cup \{\text{OUTPUT}\} \cup \{\text{AND}, \text{OR}, \text{NOT}\}
	\end{align*}
	and with the following properties:
	\begin{itemize}
		\item If $\tau(v) = \text{INPUT}$, then $v$ has no incoming edges. We call these vertices the \define{inputs gates}.
		\item If $\tau(v) = \text{OUTPUT}$, then $v$ has one incoming edge. We call these vertices the \define{output gates}.
		\item If $\tau(v) = \text{AND}$, then $v$ has two incoming edges. We call these vertices the \define{AND-gates}.
		\item If $\tau(v) = \text{OR}$, then $v$ has two incoming edges. We call these vertices the \define{OR-gates}.
		\item If $\tau(v) = \text{NOT}$, then $v$ has one incoming edge. We call these vertices the \define{NOT-gates}.
	\end{itemize}
	The inputs of $C$ are given by a tuple $(x_1, \dots, x_k)$ of distinct input gates. The output of $C$ is given by a tuple $(y_1, \dots, y_l)$ of distinct output gates.
\end{definition}

We give an example of a boolean circuit in \cref{fig:boolean_circuit_example}. Of course, we now want to use a boolean circuit to represent a function. To do this, we need to define the function computed by a boolean circuit formally.

\begin{figure}
	\centering
	\incfig{Boolean_Circuit_Example}
	\caption[Example of a Boolean Circuit]{example of a Boolean circuit with three input and four output gates.}\label{fig:boolean_circuit_example}
\end{figure}

\begin{definition}[Computed function of a boolean circuit]
	A boolean circuit $C$ with inputs $x_1, \dots, x_n$ and outputs $y_1, \dots, y_m$ computes a function $f : \bitstr^n \rightarrow \bitstr^m$ as follows:
	\begin{itemize}
		\item The input $x_i$ is assigned the value of the $i$-th bit of the argument to the function.
		\item Every other vertex $v$ is assigned the value of the gate $g$ of the vertex, applied to the values of the incoming edges of $v$.
		\item The $i$-th bit of the output of the function is the value of the output gate $y_i$.
	\end{itemize}
\end{definition}

\begin{figure}
	\centering
	\incfig{Computing_Function_Example}
	\caption[Computing a function with circuits]{example of how a function $f : {\{0, 1\}}^2 \rightarrow {\{0, 1\}}^2$ (on the top), can be computed using boolean circuits (on the bottom).}\label{fig:computing_function_example}
\end{figure}

In \cref{fig:computing_function_example}, we give an example of using a boolean circuit to compute a function, in particular for a function that is a \Tarski\ instance. From now on, we will formally represent all functions used in problems by boolean circuits.

\subsection{Complexity of boolean circuits}[Complexity of circuits]

Of course, formally, the complexity of a problem is defined in terms of the \emph{size} of the input. This means we also need to define what we mean by the size of a boolean circuit. We will use the following definition:

\begin{definition}[Size of a boolean circuit]
	The size of a boolean circuit $C$ is the number of gates in the circuit.
\end{definition}

The size of the boolean circuits is a measure of the input size, i.e.\ it gives us an indication of how many bits we need to represent the input, it also tells us how many computations are made when computing the function output. We also define the depth of a boolean circuit as follows:

\begin{definition}[Depth of a boolean circuit]
	The depth of a boolean circuit $C$ is the length of the longest path from an input gate to an output gate.
\end{definition}

The depth of a boolean circuit is a measure of the time complexity of the computation, i.e.\ it tells us how many time steps are needed to compute the output of the function. This is especially true in a parallel setting, where all gates can be seen as setting off at the same time.

%%%%%%%%%%%%%%%%%%%%%%%%%%%%%%%%%%%%%%%%%%%%%%%%%%%%%%%%%%%%%%%%%%%%%%%%%%%%%%%%%%%%%%%%%%%%%%%%%
% Subclasses of TFNP
%%%%%%%%%%%%%%%%%%%%%%%%%%%%%%%%%%%%%%%%%%%%%%%%%%%%%%%%%%%%%%%%%%%%%%%%%%%%%%%%%%%%%%%%%%%%%%%%%
\section{Subclasses of \TFNP}

Finding a complete \FNP\ problem within \TFNP\ would imply that $\NP = \coNP$~\sidecite{megiddo_total_1991}, a highly unlikely scenario. Consequently, complete problems are not expected within \TFNP, necessitating alternative approaches to investigate its structure.

\TFNP\ is a \define{semantic} class~\sidecite{papadimitriou_computational_1994}. This is known to mean that it is unlikely that we can find complete \TFNP\ problems~\sidecite{nielsen_relativization_1982}. We refer the reader to Papadimitriou's work  for a more detailed discussion of these terms and their implications. We want to explore \textit{syntactic} subclasses of \TFNP\ to address this challenge. One approach, proposed by Papadimitriou~\cite{papadimitriou_computational_1994}, categorizes search problems based on existence proofs confirming their totalness. This basic strategy leads to the detailed study of specific complexity classes discussed in the following sections.

\subsection{Polynomial Local Search (\PLS)}[\PLS]

The existence result which gives rise to \PLS\ is:
\principle{Every directed acyclic graph has a sink.}
We can then construct the class \PLS\ by defining it as all problems which reduce to finding the sink of a directed acyclic graph (DAG). Formally, we first define the problem \Localopt\ as in~\sidecite{johnson_how_1988}:

\problem{Localopt}{
\boxmarginnote{$S$ can be seen as a proposed successor, and $V$ as a potential. The goal is to find a local maxima $v$ of the potential.}
Two boolean circuits $S, V : [2^n] \rightarrow\ [2^n]$.}{A vertex $v \in [2^n]$ such that $P(S(v)) \leq\ P(v)$.}

\begin{figure}
	\centering
	\incfig{LOCALOPT_Solution_Example}
	\caption[Example of a \Localopt\ Problem]{Example of a \Localopt\ Problem with $n=3$ (8 vertices). Solid lines represent the circuit $S$. The valid solutions are colored green.}\label{fig:localopt_example}
\end{figure}

Let us discuss why solving a \Localopt\ instance is equivalent to finding the sink of a DAG\@. The circuit $S$ defines a directed graph, which might contain cycles. Only keeping the edges on which the potential increases (strictly) leads to a DAG, with as sinks exactly the $v$ such that $P(S(v)) \leq\ P(v)$ and isolated vertices which are self-loops. We give an example of a \Localopt\ instance in \cref{fig:localopt_example}. Now we can define \PLS\@:

\begin{definition}[Polynomial Local Search (\PLS)]
	The class \PLS\ is the set of all \TFNP\ problems that reduce to \Localopt.
\end{definition}

Studying ``simple'' problems such as \Localopt\ is particularly insightful because it seems difficult to find a better algorithm than simply traversing the graph, which solves these problems. One should emphasize that there very well might be an algorithm which outperforms a traversal of the graph. For instance, there could be a clever way of analyzing the input circuit, which leads to a faster algorithm. Such an algorithm has yet to be discovered, and it is seen as highly unlikely that it exists. Of course --- and here lies the difficulty of complexity theory --- we do not know how to prove that such an algorithm cannot exist. Hence it seems sound to conjecture that \PLS\ is not in \FP\@.

\subsection{Polynomial Parity Argument on Directed Graphs (\PPAD)}[\PPAD]

Now we want to discuss the complexity class \PPAD, introduced by Papadimitriou~\sidecite{papadimitriou_complexity_1994} as one of the first syntactic subclasses of \TFNP\@. The existence result giving rise to this class is: \principle{If a directed graph has an unbalanced vertex, then it has at least one other unbalanced vertex.}
\PPAD\ can be defined using the problem \textsc{End-of-Line} as introduced in~\sidecite{daskalakis_complexity_2009}.

\problem{End-of-Line (\textsc{EoL})}{
A boolean circuit $S, P : \bitstr^n \rightarrow \bitstr^n$ such that $P(0^n) = 0^n \neq S(0^n)$ ($0^n$ is a source.)
}{
\boxmarginnote{Here, $S$ can be thought of as giving the successor of a vertex, and $P$ as giving the predecessor of a vertex.}
An $x \in \bitstr^n$ such that either:
\begin{itemize}
	\itemspecial{S1} $P(S(x)) \neq x$ ($x$ is a sink) or
	\itemspecial{S2} $S(P(x)) \neq x \neq 0^n$ ($x$ is a non non-standard source)
\end{itemize}
}

\begin{figure}[ht]
	\centering
	\incfig{PPAD_Example}
	\caption[Example of an \textsc{End-of-Line} Problem]{Example of an \textsc{End-of-Line} Problem with $n=4$ (16 vertices). Solid lines represent the circuit $S$, and dashed lines represent the circuit $P$. The solutions \textbf{(S1)} are the sinks $x=5$, $x=9$ and $x=11$. The solutions \textbf{(S2)} are the non-standard sources $x=6$ and $x=10$.}\label{fig:ppad_example}
\end{figure}

These boolean circuits represent a directed graph with maximal in and out-degree of one by having an edge from $x$ to $y$ if and only if $S(x) = y$ and $P(y) =x$. The goal is to find a sink in the graph or another source\marginnote{Notice that \textsc{End-of-Line} allows cycles and that these do not induce solutions.}. It can be shown that the general case of finding a second imbalanced vertex in a directed graph (a problem called \textsc{Imbalance}) can be reduced to \textsc{End-of-Line} and is \PPAD-complete~\sidecite{goldberg_hairy_2021}. Now we can define the complexity class \PPAD\ as follows:

\begin{definition}[\PPAD]
	The class \PPAD\ is the set of all \TFNP\ problems that reduce to \textsc{End-of-Line}.
\end{definition}

\subsection{End of Potential Line (\EOPL)}[\EOPL]

Next, we discuss the complexity class \EOPL{} introduced in~\sidecite{EOPL_introduction}. The existence results giving rise to \EOPL\ is:
\principle{In a directed acyclic graph, there must be at least two unbalanced vertices.}
Similarly to \PLS, acyclicity will be enforced using a potential.

\problem{End of Potential Line}{
Two boolean circuits $S, P : \bitstr^n \rightarrow \bitstr^n$, and a boolean circuit $V : \bitstr^n \rightarrow [2^n - 1]$, such that $0^n$ is a source, (i.e.\ $P(0^n) = 0^n \neq S(0^n)$).
}{
An $x \in \bitstr^n$ such that either:
\begin{itemize}
	\itemspecial{S1} $P(S(x)) \neq x$ ($x$ is a sink)
	\itemspecial{S2} $S(P(x)) \neq x \neq 0^n$ ($x$ is a \define{non-standard source})
	\itemspecial{V1} $S(x) \neq x$, $P(S(x)) = x$ and $V(S(x)) \leq V(x)$ (violation of the monoticity of the potential)
\end{itemize}
}
\marginnote[-50mm]{Here, $S$ can be thought of as giving the successor of a vertex, and $P$ as giving the predecessor of a vertex.
	$ V $ is a potential that is supposed to increase monotonously along the line.}

\begin{figure}[ht]
	\centering
	\incfig{EOPL_Solution_Example}
	\caption[Example of an \EOPL\ Problem]{Example of an \textsc{End of Potential Line} Problem with $n=4$ (16 vertices). Solid lines represent the circuit $S$, and dashed lines represent the circuit $P$. The solutions \textbf{(S1)} are the sinks $x=5$, $x=9$ and $x=11$. The solutions \textbf{(S2)} are the non-standard sources $x=6$ and $x=10$. The vertices $2$ and $15$ are violations of the potential \textbf{(V1)}.}\label{fig:eopl_example}
\end{figure}

$S$ and $P$ can be thought of as representing a directed line. Finding another source (a non-standard source) is a solution, as it is the end of another line. The potential serves as a guarantee of acyclicity. We note that this is the main difference between \EndOfLine\ and \EndOfPotentialLine, as cycles can arise in the former, and do not yield solutions (or violations), while a cycle implies a violation of the potential in the latter. Now, we can define the complexity class \EOPL\@.

\begin{definition}[\EOPL]
	The class \EOPL\ is the set of all \TFNP\ problems that reduce to \textsc{End of Potential Line}.
\end{definition}

\subsection{Unique End of Potential Line}[\UEOPL]\label{sec:ueopl}

We introduce a variation of the \EndOfPotentialLine\ problem, where we only allow one line. This problem, called \UniqueEndOfPotentialLine, and first introduced in~\sidecite{fearnley_unique_2020} is constructed like \EndOfPotentialLine, but with an additional constraint which only allows one line:

\problem{Unique End of Potential Line}{
Two boolean circuits $S, P : \bitstr^n \rightarrow \bitstr^n$, and a boolean circuit $V : \bitstr^n \rightarrow [2^n - 1]$, such that $0^n$ is a source, (i.e.\ $P(0^n) = 0^n \neq S(0^n)$).
}{
Either:
\begin{itemize}
	\itemspecial{S1} An $x \in \bitstr^n$ such that $P(S(x)) \neq x$ ($x$ is a sink);
	\itemspecial{V1} An $x \in \bitstr^n$ such that $S(P(x)) \neq x \neq 0^n$ ($x$ is a non-standard source);
	\itemspecial{V2} An $x \in \bitstr^n$ such that $S(x) \neq x$, $P(S(x)) = x$ and $V(S(x)) \leq V(x)$ (violation of the monotonicity of the potential);
	\itemspecial{V3} Two points $x, y \in \bitstr^n$ such that $x \neq y$, $x \neq S(x)$, $y \neq S(y)$ and either $V(x) = V(y)$ or $V(x) < V(y) < V(S(x))$. (two lines)
\end{itemize}
}

We note that the solutions of type \textbf{(V1)}, which were not violations in the \EndOfPotentialLine\ problem, are now violations in the \UniqueEndOfPotentialLine\ problem, as finding a second (non-standard source), implies that there is a second line. Now, one might ask why the third violation is necessary. We add the third violation to make it easier to show that there are two lines. We might find two points $x, y$ on two different lines that are exponentially far away from the start of their respective lines, making it hard to find the violation. Adding the violation \textbf{(V3)} ensures that all points that are provably on two different lines are a violation. Indeed, we have:
\begin{itemize}
	\item If $V(x) = V(y)$, then if $x$ and $y$ are on the same line, they must be the same point, as the potential is strictly increasing along the line.
	\item If $V(x) < V(y) < V(S(x))$ and $x$ and $y$ are on the same line, then $y$ cannot lie before $x$ on the line, as $V(y) > V(x)$, but it can also not lie after $S(x)$, as $V(y) < V(S(x))$. Hence, $x$ and $y$ must be on different lines.
\end{itemize}

\begin{figure}[ht]
	\centering
	\incfig{UEOPL_Solution_Example}
	\caption[Example of a \UEOPL\ Problem]{Example of a \UniqueEndOfPotentialLine\@. Solid lines represent the circuit $S$, and dashed lines represent the circuit $P$. The solution is the sink \textbf{(S1)} $x=5$. We have a non-standard source \textbf{(V1)} at $x=6$. We have a violation of monoticity \textbf{(V2)} at $x=2$. We have violations \textbf{(V3)} formed by $x=4$, $y=6$ and by $x=7$, $y=5$}\label{fig:ueopl_example}
\end{figure}

We give an example of a \UniqueEndOfPotentialLine-instance in \cref{fig:ueopl_example}. We can now define the complexity class \UEOPL\ analogously as we defined \EOPL\ previously.

\begin{definition}[\UEOPL]
	The class \UEOPL\ is the set of all \TFNP\ problems that reduce to \textsc{Unique End of Potential Line}.
\end{definition}

We note that \UEOPL\ is a subclass of \EOPL, as the constraint of having only one line is more restrictive than the constraint of having at most one line. It is conjecture that \UEOPL\ is a strict subclass of \EOPL, but this has yet to be proven \sidecite{fearnley_unique_2020}.

\subsection{The classes \PPADS\ and \SOPL}[\PPADS\ and \SOPL]\label{sec:ppads_and_sopl}

We introduce variations slight variantions of the classes \PPAD\ and \EOPL\ in this section. Recall that \EndOfLine\ and \EndOfPotentialLine\ are respective complete problems for \PPAD\ and \EOPL\@. If instead of finding any end of the line we only allow sinks we get new problems \SinkOfPotentialLine\ and \SinkOfLine\@:

\problem{SinkOfLine}{
Boolean circuits $S, P : \bitstr^n \rightarrow \bitstr^n$ such that $P(0^n) = 0^n \neq S(0^n)$.
}{
An $x \in \bitstr^n$ such that $P(S(x)) \neq x$ (a sink).
}

\problem{SinkOfPotentialLine}{
Boolean circuits $S, P : \bitstr^n \rightarrow \bitstr^n$ and $V : \bitstr^n \rightarrow [2^n - 1]$ such that $P(0^n) = 0^n \neq S(0^n)$.
}{An $x \in \bitstr^n$ such that either:
\begin{itemize}
	\itemspecial{S1} $P(S(x)) \neq x$ ($x$ is a sink), or;
	\itemspecial{V1} $S(x) \neq x$, $P(S(x)) = x$ and $V(S(x)) \leq V(x)$ (violation of the monoticity of the potential).
\end{itemize}
}

The classes \PPADS\ and \SOPL\ are then constructed so that \SinkOfPotentialLine\ and \SinkOfLine\ are complete problems.
\begin{definition}[\PPADS]
	The class \PPADS\ is the set of all \TFNP\ problems that reduce to \textsc{Sink of Potential Line}.
\end{definition}
\begin{definition}[\SOPL]
	The class \SOPL\ is the set of all \TFNP\ problems that reduce to \textsc{Sink of Line}.
\end{definition}

These two classes are not directly relevant for this thesis, but they are a stepping used in the recent collapse of the \TFNP\ hierarchy~\sidecite{goos_further_2022}, which we will discuss in \cref{ch:location_tarski}. For the time beeing it suffice to know that these classes are weaker than \PPAD\ and \EOPL\@. That is:
\begin{align*}
	\PPAD \subseteq \PPADS \quad \text{and} \quad \EOPL \subseteq \SOPL.
\end{align*}
This is because finding a sink is more restrictive than finding any end of the line.

Now we have introduced an defined the main subclasses of \TFNP, which are relevant to the work in this thesis. We refer the reader to~\sidecite{hollender_structural_2021} for a complete overview of the subclasses of \TFNP\@. We will discuss some of the known collapse inside the \TFNP\ hierarchy in \cref{ch:location_tarski}. We now move on to the main \TFNP-problem we study in this thesis, the \Tarski\ problem, which we introduce in the following section.

%%%%%%%%%%%%%%%%%%%%%%%%%%%%%%%%%%%%%%%%%%%%%%%%%%%%%%%%%%%%%%%%%%%%%%%%%%%%%%%%%%%%%%%%%%%%%%%%%
% The Tarski Problem
%%%%%%%%%%%%%%%%%%%%%%%%%%%%%%%%%%%%%%%%%%%%%%%%%%%%%%%%%%%%%%%%%%%%%%%%%%%%%%%%%%%%%%%%%%%%%%%%%
\section{The \Tarski\ Problem}[\Tarski\ Problem]\label{sec:tarski_problem}

\subsection{Definition of the \Tarski\ Problem}[\Tarski\ Definition]

Next, we introduce the \Tarski\ Problem. Before we do this, we recall that there is a partial order on the $d$ dimensional lattice ${[N]}^d$, given by $x \leq y$ if and only if $x[i] \leq y[i]$\marginnote{We denote by $x[i]$ the value of the $i$-th coordinate of $x$.} for all $i \in \{1, \dots, d\}$\marginnote{Notice that $x \not\leq y$ does \emph{not} imply $x \geq y$. In particular, two points are not always comparable.}. We can now define functions on this lattice, and in particular, we can define monotone functions.

\begin{definition}[Monotone function]
	\boxmarginnote{Such functions are also called \define{order preserving} functions in the litterature.}
	A function $f : {[N]}^d \rightarrow {[N]}^d$ is \define{monotone} if for all $x, y \in {[N]}^d$ we have $x \leq y$ implies $f(x) \leq f(y)$.
\end{definition}

The \Tarski-problem originates from Tarski's fixed point Theorem, introduced in~\sidecite[-5mm]{tarski_lattice-theoretical_1955}.

\begin{theorem}[Tarski's fixed point Theorem]
	Let $L$ be a complete lattice and $\leq$ be a partial order on $L$\boxmarginnote{This Theorem is also known as the Knaster–Tarski Theorem in the literature.}. Furthermore, let $f$ be a monotone function with respect to $\leq$. Then the set of fixed points of $f$ in $L$ forms a complete lattice under $\leq$.
\end{theorem}

A proof of this Theorem can be found in the previously mentioned work~\cite{tarski_lattice-theoretical_1955}.In our setting we will only use a special corollary of this theorem which can be formulated as follows:

\begin{corollary}[Simplified version of Tarski's theorem]
	Let $f : {[N]}^d \rightarrow {[N]}^d$ be a monotone function on the $d$-dimensional lattice. If $f$ is monotone (with respect to the previously discussed partial order), then $f$ has a fixed point, i.e.\ there is an $x^* \in {[N]}^d$ such that $f(x^*)=x^*$.
\end{corollary}

From now on, we will refer to this corollary when referring to \emph{Tarski's fixed point theorem}. A proof of this special case follows from how \cref{alg:iterative_tarski_solver}, which we will discuss shortly, finds a fixed-point of such a function.

Without surprise, the \Tarski\ problem, defined in~\sidecite{etessami_tarskis_2020}, is now to find such a fixed point. Formally, we define the problem as follows:
\problem{Tarski}{A boolean circuit $f : {[N]}^d \rightarrow {[N]}^d$.}{Either:
\begin{itemize}
	\itemspecial{S1} An $x \in {[N]}^d$ such that $f(x)=x$ (fixed point) or
	\itemspecial{V1} $x, y \in {[N]}^d$ such that $x \leq y$ and $f(x) \nleq f(y)$ (violation of monotonicity).
\end{itemize}
}

\begin{figure}
	\centering
	\incfig{Tarski_Solution_Example}
	\caption[Example of a \Tarski\ instance]{Example of a two-dimensional \Tarski\ instance. A fixed point is located at $x = (4, 5)$. The path to the fixed point which \cref{alg:iterative_tarski_solver} finds is colored in green.}\label{fig:tarski_example}
\end{figure}

This is a total search problem, as there will always either be a fixed point or a point violating monotonicity. We give an example of a two-dimensional \Tarski\ instance in \cref{fig:tarski_example}. Before we discuss the location of \Tarski\ in the \TFNP\ landscape and two known algorithms for solving \Tarski, we want to discuss a useful lemma, which allows us to simplify the study of \Tarski\ instances. The definition of \Tarski\ instances allows for the image of a point to be located anywhere in the lattice; we will show that we can reduce to the cases where the image of a point is in the immediate neighborhood of the point.

\begin{lemma}[Simplyfying \Tarski]
	Let $f : {[2^n]}^d \rightarrow {[2^n]}^d$ be a \Tarski\ instance on a complete lattice ${[2^n]}^d$. Consider $\ftilde : {[2^n]}^d \rightarrow {[2^n]}^d$ given by: \boxmarginnote{Notice that given a circuit $C$ which computes $f$, we can construct a circuit $\Ctilde$ which computes $\ftilde$ by adding $\BigO{\poly{(n,d)}}$ gates to $C$. This means that both problems are equivalent in terms of complexity.}
	\begin{align*}
		\ftilde(x)[i] = \begin{cases}
			                x[i] + 1 & \text{ if } f(x)[i] > x[i], \\
			                x[i]     & \text{ if } f(x)[i] = x[i], \\
			                x[i] - 1 & \text{ if } f(x)[i] < x[i].
		                \end{cases} \quad \text{for all $i \in \set{1, \dots, d}$}
	\end{align*}
	Then, for any two points $x, y \in {[2^n]}^d$, $f(x) \leq f(y)$ if and only if $\ftilde(x) \leq \ftilde(y)$.
\end{lemma}
\begin{proof}
	The lemma follows directly by observing that for comparable $x, y$ and for all, $i \in \set{1, \dots, d}$ we have: $f(x)[i] \leq f(y)[i]$ if and only if $\ftilde(x)[i] \leq \ftilde(y)[i]$.
\end{proof}
This means that in this thesis, we can consider the simplified version of the \Tarski\ problem, where for every $x \in {[2^n]}^d$, we have $\norminf{x - f(x)} \leq 1$, which we will implicitly assume from now on.

\subsection{Algorithms for solving \Tarski}[\Tarski\ Algorithms]\label{sec:tarski_algorithms}

We briefly discuss the most common algorithms for solving \Tarski\ instances. We begin with a straightforward algorithm, which is based on the following observation:
\begin{remark}
	Let $f$ be a \Tarski\ instance on a complete lattice $L$. If $f$ is monotone, and for some $x \in L$ we have $f(x) \geq x$, then $f(f(x)) \geq f(x)$.
\end{remark}
Now, note that by starting at the point $\mathbf{0} = 0^d$ and iterating the function $f$, we will eventually reach a fixed point. This means that we can construct an iterative algorithm for solving \Tarski, as described in \cref{alg:iterative_tarski_solver}.

\begin{algorithm}
	\caption{Iterative Algorithm for \Tarski}\label{alg:iterative_tarski_solver}
	\KwData{A boolean circuit $f : L \rightarrow L$}
	\KwResult{A fixed point of $f$}
	$x \leftarrow \mathbf{0}$ \;
	\While{$f(x) \neq x$}{
		\If{$f(x) \not\geq x$}{
			\Return{``$x_{\text{old}}, x$ are a violation of monotonicity.'' \;}
		}
		\Else{
			$x_{\text{old}} \leftarrow x$ \;
			$x \leftarrow f(x)$ \;
		}
	}
	\Return{$x$ \;}
\end{algorithm}

When discussing the runtime of the algorithms, we present we will be discussing the \define{query complexity} of these algorithms, i.e.\ the number of times we evaluate $f$ on a lattice point. The path which \cref{alg:iterative_tarski_solver} takes to solve a \Tarski\ instance is colored green in \cref{fig:tarski_example}. While \cref{alg:iterative_tarski_solver} might not be very efficient --- it runs in worst-case time $\BigO{d \cdot N}$ for $L = {[N]}^d$ --- it does have some theoretical applications for locating \Tarski\ inside \TFNP\@. Previous work~\sidecite{etessami_tarskis_2020} showed that \Tarski\ lies in \PLS\ by considering the set of possible states of the previously described algorithm, together with a potential function given by $V(x) = \sum_{i=1}^{d}{x[i]}$, and showing that this potential is monotone along the states of the algorithm. The circuit $S$ associates the state of the algorithm to the next state it will be in.

Next, we describe a more advanced algorithm, due to~\sidecite{dang_computations_2020}, for solving \Tarski\ instances and also give an alternative presentation and simplified proof of its correctness. The following notation aims to make the argument as clear as possible. For a given complete lattice $L = [N_1] \times \cdots \times [N_d]$ and some vertex $x \in L$ we define the following sublattices:
\marginnote{Recall that we define ${[m] = \set{0, \dots, m-1}}$ for $m \in \N$.}
\begin{align*}
	L_{\leq x} & = [x[1]+1] \times \cdots \times [x[d]+1],                          \\
	L_{\geq x} & = \winterval{x[1]}{N_1} \times \cdots \times \winterval{x[d]}{N_d}
\end{align*}
$L_{\leq x}$ are exactly the points of the lattice that are smaller than $x$, and $L_{\geq x}$ are exactly the points of the lattice that are larger than $x$.\marginnote[-10mm]{We denote by $\winterval{a}{b}$ the set of whole numbers $\{a, a+1, \dots, b\}$.} For a given dimension $k \in \set{1, \dots, d}$ and $K \in [N_k]$, we also define the following sublattices:
\begin{align*}
	L_{k < K} & = [N_1] \times \cdots \times [N_{k-1}] \times [K] \times [N_{k+1}] \times \cdots \times [N_d],                 \\
	L_{k = K} & = [N_1] \times \cdots \times [N_{k-1}] \times \{K\} \times [N_{k+1}] \times \cdots \times [N_d],               \\
	L_{k > K} & = [N_1] \times \cdots \times [N_{k-1}] \times \{K+1, \dots, N_k\} \times [N_{k+1}] \times \cdots \times [N_d].
\end{align*}
The algorithm --- and in particular, our proof of the correctness --- is based on the following observation:
\begin{remark}[Progress points induce \Tarski-instances]\label{rem:progress_points_and_subsinstances}
	Let $L = [N_1] \times \cdots \times [N_d]$ be a complete lattice and $f : L \rightarrow L$ a monotone function. Then:
	\begin{enumerate}
		\item If for some $x \in L$ we have $f(x) \leq x$, then $f$ has a fixed point in $L_{\leq x}$.
		\item If for some $x \in L$ we have $f(x) \geq x$, then $f$ has a fixed point in $L_{\geq x}$.
	\end{enumerate}
\end{remark}
\begin{proof}
	Let $x \in L$ such that $f(x) \leq x$. Then for all $y \in L_{\leq x}$ we have $y \leq x$ and hence $f(y) \leq f(x) \leq x$, which shows that $f$ is a \Tarski\ instance on $L_{\leq x}$. By Tarski's fixed point Theorem, $f$ has a fixed point in $L_{\leq x}$. The proof for the second point is analogous.
\end{proof}
Hence, points with these properties seem particularly interesting when searching for fixed points of $f$, which is why we want to give them a name:
\begin{definition}[Progress point]
	Let $f: L \rightarrow L$ be a \Tarski-instance. A point $x \in L$ is called a \define{forward point} if $f(x) \geq x$. A point $x \in L$ is called a \define{backward point} if $f(x) \leq x$. Finally, a point $x \in L$ is called a \emph{progress point} if it is either a forward or a backward point.\boxmarginnote{The lattice's smallest vertex and largest vertex are always progress points.}
\end{definition}
This means that if we have a progress point, we can reduce the area where we need to search for a fixed point. The question now becomes: how do we find such an $x$? The algorithm we will present is based on the following observation:
\begin{remark}
	Let $f : L \rightarrow L$ on a complete lattice ${L = [N_1] \times \cdots \times [N_d]}$, for a monotone function $f$, be a \Tarski\ instance. By fixing some dimension ${k \in \set{1, \dots, d}}$, we can define the function ${f_{k=K} : L_{k=K} \rightarrow L_{K=k}}$ as follows:
	\begin{align*}
		f_{k=K}(x)[i] = \begin{cases}
			                f(x)[i] & \text{ if } i \neq k, \\
			                K       & \text{ if } i = k.
		                \end{cases} \quad \text{for all $i \in \set{1, \dots, d}$}
	\end{align*}
	Then $f_{k=K}$ is a monotone \Tarski\ instance on $L_{k=K}$, and if $x^*$ is a fixed point of $f_{k=K}$, then $x^*$ is a progress point of $f$.
\end{remark}
This means we if we know how to solve a $d-1$ dimensional \Tarski\ instance, then we can find a point $x$ such that $f(x) \geq x$ or $f(x) \leq x$. We will be able to leverage this to construct an algorithm, once we have proven this remark.
\begin{proof}
	The monotonicity of $f_{k=K}$ follows directly from the monotonicity of $f$. \par
	The fact that $x^*$ is a progress point follows from the fact that if $x^*$ is a fixed point of $f_{k=K}$, then $f(x^*)[i] = x^*[i]$, for all $i \neq k$. This means that if $f(x^*)[k] \leq x^*[k]$, then $f(x^*) \leq x^*$ and if $f(x^*)[k] \geq x^*[k]$, then $f(x^*) \geq x^*$.
\end{proof}
By choosing, $K = \lfloor \frac{N_k}{2} \rfloor$ we can find a progress point $x$ such that both $L_{\leq x}$, and $L_{\geq x}$ have at most half the size of $L$. This means we can reduce the search space by a factor of at least two by solving a $d-1$ dimensional \Tarski\ instance. We can solve a $d$ dimensional \Tarski\ instance by repeatedly solving $d-1$ dimensional \Tarski\ instances and reducing the search space size by a factor of at least 2 in each step. This means we can solve a $d$ dimensional \Tarski\ instance by combining a $d-1$ dimensional \Tarski\ solver and a binary search. The $d-1$ dimensional instances can be solved recursively. We give the recursive algorithm for solving \Tarski\ instances in \cref{alg:recursive_tarski_solver}.
\begin{algorithm}
	\caption{Recursive Algorithm for \Tarski}\label{alg:recursive_tarski_solver}
	\SetKwFunction{RecursiveTarskiSolver}{RecursiveTarskiSolver}
	\SetKwProg{Fn}{Function}{:}{}
	\Fn{\RecursiveTarskiSolver{$f : L \rightarrow L$, $d$}}{
		\tcc{Binary search in the $d$-th dimension}
		\Let{l}{0}, \Let{r}{$N_d$} \; \tcc*[r]{The search space is $\lbrack l, r \rbrack$}
		\While{$r - l > 1$}{
			\Let{$m$}{$\floor{\frac{l + r}{2}}$} \tcc*[r]{Middle of the interval}
			\If{$d = 1$}{
				\Let{$x^*$}{$m$}
			}
			\Else{
				\tcc{Solve the $d-1$ dimensional instance}
				\Let{$x^*$}{\RecursiveTarskiSolver{$f_{d=m}, d-1$}} \;
			}
			\If{$f(x^*)[d] \leq x^*[d]$}{
				\Let{r}{$m$}
			}
			\Else{
				\Let{l}{$m$}
			}
		}
		\Return{$x^*$}
	}
\end{algorithm}

A simple analysis shows that this algorithm runs with query complexity $\BigO{\log^{d}{N}}$, where $N$ is the side length of the lattice. It was conjectured that this is an asymptotically optimal algorithm for \Tarski~\sidecite{etessami_tarskis_2020}. This turned out not to be true, as a better algorithm was developed~\sidecite{fearnley_faster_2022}, which mostly relies on a faster way of finding a progress point in the three-dimensional case, which they call the \emph{inner algorithm}. An \emph{outer algorithm} then repeatedly applies the inner algorithm on 3-dimensional instances. Overall, this approach achieves a query complexity of $\BigO{\log^{\ceil{\frac{2d}{3}}}{N}}$ for $L = {[N]}^d$\marginnote{The authors claim a bound of $\BigO{\log^{2\ceil{\frac{d}{3}}}{N}}$ in the paper, but this seems to be a mistake, as argued in~\cite{chen_improved_2022}.}.

A further advance using more advanced methods and assuming that the number of dimensions $d$ is constant were made in~\sidecite{chen_improved_2022}. They achieve a query complexity of $\BigO{\log^{\ceil{\frac{d+1}{2}}}{N}}$ for $L = {[N]}^d$. This is to date the best known bound for solving \Tarski\ instances.

\begin{table}
	\centering
	\renewcommand{\arraystretch}{1.5}
	\begin{tabular}{@{}l r@{}}
		\toprule
		\textbf{Algorithm}                 & \textbf{Query complexity}               \\
		\midrule
		\cref{alg:iterative_tarski_solver} & $\BigO{d \cdot N}$                      \\
		\cref{alg:recursive_tarski_solver} & $\BigO{\log^{d}{N}}$                    \\
		\textcite{fearnley_faster_2022}    & $\BigO{\log^{\ceil{\frac{2d}{3}}}{N}}$  \\
		\textcite{chen_improved_2022}      & $\BigO{\log^{\ceil{\frac{d+1}{2}}}{N}}$ \\
		\bottomrule
	\end{tabular}
	\caption[Comparison of known algorithms for solving \Tarski]{Known algorithms for solving \Tarski\ instances and their query complexity.}\label{tab:tarski_algorithms}
\end{table}

\subsection{Lower bounds for \Tarski}

The best-known lower bounds for \Tarski\ are given by~\sidecite{etessami_tarskis_2020}. They showed that in the black-box model, where the only way to access the function $f$ is by querying it, solving a $d$-dimensional \Tarski\ requires solving at least $\Omega(\log{N})$ one-dimensional \Tarski\ instances, which are as difficult as binary search, hence this means that solving a $d$-dimensional \Tarski\ instance requires at least $\Omega(\log^{2}{N})$ queries. This means that the upper and lower bounds are equal in the two- and three-dimensional cases, but in all other cases, there remains a gap. In particular, the best-known lower bound for solving \Tarski\ does not depend on the dimension $d$, which seems somewhat unexpected.

This gives us reason to study \Tarski\ under the lens of complexity theory, in particular to understand where \Tarski\ lies in the \TFNP\ landscape, as we will discuss in \cref{ch:location_tarski}.

\subsection{Variants of \Tarski}[\Tarski\ Variants]

Before we conclude this chapter, we want to discuss some variants of the \Tarski\ problem. The first variant we introduce is the \emph{promise} variant of \Tarski, which is defined as follows:

\problem{PromiseTarski}{A boolean circuit $f : {[N]}^d \rightarrow {[N]}^d$ such that ${x \leq y \implies f(x) \leq f(y)}$}{A fixpoint $x$ of $f$.}

Notably, in this variant of \Tarski\ we are \emph{promised} that we have a monotone function, and hence we do not allow violations of monotonicity as solutions (as these should not be possible).

Next we introduce two variants of \Tarski\ which are also promise variants of \Tarski. On top of promising that the function is monotone, we are promised some properties related to uniqueness of solutions. We note that these problems can be turned into total problems, by adding violations, as discussed in \cref{sec:promise_problems}. For the sake of simplicity, we introduce them in their promise variants. The first variant we introduce is the \emph{unique fixpoint} variant of \Tarski, which is defined as follows:

\problem{PromiseUniqueTarski}{A boolean circuit $f : {[N]}^d \rightarrow {[N]}^d$ such that $f$ is monotone and there is a unique fixpoint $x$}{The unique fixpoint $x$ of $f$.}

A very notable recent result is that the query complexity of \SuperUniqueTarski\, \UniqueTarski\ is equal to the query complexity of \Tarski~\sidecite{chen_reducing_2023}.  We also introduce an even stronger uniqueness variant of \Tarski\ next. This time the promise is that when fixing a subset of coordinates, the induced function has a unique fixpoint, i.e.\ $\ftilde = f_{d_1=K_1, \dots, d_k=K_k}$ has a unique fixpoint for any choice of $d_1, \dots, d_k \in \set{1, \dots, d}$ and $K_1, \dots, K_k \in [N]$.

\problem{PromiseSuperUniqueTarski}{A boolean circuit $f : {[N]}^d \rightarrow {[N]}^d$ such that $f$ is monotone and when fixing a subset of coordinates the induce function $\ftilde$ has a unique fixpoint.}{The unique fixpoint $x$ of $f$.}

The problem \SuperUniqueTarski\ is a very strong promise variant of \Tarski, it implies that \cref{alg:recursive_tarski_solver} finds a unique fixpoint in every step.