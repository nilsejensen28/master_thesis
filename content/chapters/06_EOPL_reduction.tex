\setchapterpreamble[u]{\margintoc}
\chapter{Reducing \Tarski\ to \EOPL}\label{ch:eopl_reduction}

In the previous chapter, we demonstrated how one can prove the membership of \Tarski\ in \PPAD\ through a reduction to \Sperner\ and then a reduction from \Sperner\ to \EndOfLine. We want to argue that the \EndOfLine\ instance we obtain when reducing a monotone function $f$ is almost an \EndOfPotentialLine\ instance. In particular, recall from \cref{ch:preliminaries} that the main difference is the absence of cycles in \EndOfPotentialLine\ instances, while such cycles might exist in \EndOfLine\ instances. The absence of cycles in the \EndOfPotentialLine\ instances is guaranteed by a potential. In this chapter, we will argue that the \EndOfLine\ instances that we obtain when reducing monotone functions have no cycles.

In order to argue this result, we will need a more meticulous examination of the structure of a \Tarski\ instance and the induced coloring of the lattice points. We will start by arguing that there are no cycles in the two-dimensional case. Doing this will motivate the following steps to generalize the results to higher dimensions.

First, we will construct a specific simplicial decomposition of the lattice. We do this to obtain specific valuable properties. Then, we will discuss how we orient simplices, their faces, and ultimately, the colored faces we traverse when reducing \Sperner\ to \EndOfLine. Finally, we will discuss how these elements interplay to guarantee the absence of cycles in three dimensions and how this can be generalized.

%%%%%%%%%%%%%%%%%%%%%%%%%%%%%%%%%%%%%%%%%%%%%%%%%%%%%%%%%%%%%%%%%%%%%%%%%%%%%%%%%%%%%%%%%%%%%%%%%
%No cycles in End of Line instances
%%%%%%%%%%%%%%%%%%%%%%%%%%%%%%%%%%%%%%%%%%%%%%%%%%%%%%%%%%%%%%%%%%%%%%%%%%%%%%%%%%%%%%%%%%%%%%%%%
\section{No cycles in the \EndOfLine\ instance}[No cycles in the \EndOfLine\ instance]

\begin{remark}
	Assume that we have a valid colored simplicial sequence that crosses a hyperplane obtained by fixing one dimension $H = L_{k = K}$, for a dimension $k \in \set{1, \dots, d}$ and $K \in [N]$, twice. Let $F_i$ and $F_j$ be the transition faces which cross the hyperplane. Then: \boxmarginnote{This means that when looking at the hyperplane from any side, $F_i$ and $F_j$ have opposite orientations.}
	\begin{align*}
		\orient_{S_i}(F_i) = - \orient_{S_{j+1}}(F_j)
	\end{align*}
\end{remark}
\begin{proof}
	This is a reformulation of the more general \cref{lem:consistent_orientation_of_transition_faces}.
\end{proof}

Now, we are ready to prove our main result. We will start by proving that there are no cycles in the three-dimensional case and then extend this to the general case.

\begin{theorem}[No cycles in three-dimensional \Tarskistar]\label{thm:no_cycles_in_three_dimensional_tarskistar}
	There are no cycles in the \EndOfLine\ instance which we reduce three dimensional \Tarskistar\ instances to.
\end{theorem}

Before we formally prove this theorem, we want to give the inuition for the proof. Assume we have a cycle of simplices which for three cycles of faces spanned by the vertices colored 0 and 1, 1 and 2, and 2 and 0. The idea is to show that we have too many contraints for this to be possible. In particular we have the following constraints some of which we have already proven, and some of which we will prove shortly.
\begin{enumerate}
	\item We must move in all three dimensions.
	\item The orientation of the faces spanned by the vertices colored 0 and 1 must be consistent.
	\item The orientation of the faces spanned by the vertices colored 1 and 2 must be consistent.
	\item The orientation of the faces spanned by the vertices colored 2 and 0 must be consistent.
	\item The orientation of the transition faces between these faces must be consistent.
	\item When crossing a hyperplane with fixed dimension 1, the transition faces must have opposite orientations when looking at the hyperplane from any side, and inside these faces:
	      \begin{enumerate}
		      \item The vertex colored 0 must be larger than the vertex colored 1;
		      \item The vertex colored 2 must be larger than the vertex colored 0;
		      \item The vertex colored 0 must appear after $\xrightarrow{3}$.
	      \end{enumerate}
	\item When crossing a hyperplane with fixed dimension 2, the transition faces must have opposite orientations when looking at the hyperplane from any side, and inside these faces:
	      \begin{enumerate}
		      \item The vertex colored 0 must be larger than the vertex colored 2;
		      \item The vertex colored 0 must appear after $\xrightarrow{3}$.
	      \end{enumerate}
	\item When crossing a hyperplane with fixed dimension 3, the transition faces must have opposite orientations when looking at the hyperplane from any side.
\end{enumerate}
Now the whole idea of the proof is to show that the combination of these constraints is not possible. In particular, if you picture the \emph{tube} formed by these simplices, it cannot \emph{twist} completely because of these constraints, in particular because the vertex colored $0$ must be on the upper side in dimension 3, of every cell.

We are ready for the formal proof, which we will divide into smaller claims. We start with an easier case, in which a hypothetical cycle only moves in two dimensions. First we want to shortly discuss what we mean by moving in a dimension.

\begin{definition}[Moving in a dimension]
	We say that a sequences of simplices ${\left(S_i\right)}_{i=1}^k$ moves in a dimension $l$ if it passes through two cells, which differ in dimension $l$.
\end{definition}

This is equivalent to saying that the transition faces $F_i$ is missing dimension $l$ for some $i$. Now let us show that a cycle must move in at least three dimensions.

\begin{claim}[No simple cycles in $d$-dimensional \Tarskistar-instances]\label{claim:no_simple_cycles}
	Assume that ${\left(S_i\right)}_{i=1}^k$ is a cycle of simplices for colors $C = \set{0, 1, \dots, d-1}$. Then this cycle must move in at least three dimensions.
\end{claim}
\begin{proof}
	A cycle most move in at least one dimension, because otherwise it would not be a cycle. Assume for the sake of contradiction that the cycle only moves in two dimensions. Then we must have that the cycle moves in dimension $q$ and $l$. At least $q$ or $l$ must be in $C$. Assume without loss of generality that $q \in C$. Then the cycle must cross a hyperplane $H$ obtained by fixing dimension $q$ at least twice. Let $F_i$ and $F_j$ be the transition faces which cross the hyperplane $H$.

	The $d-1$ dimensional cells of $H$ which contain $F_i$ and $F_j$ differ only in dimension $l$. Assume without loss of generality that $F_i$ is smaller than $F_j$ in dimension $l$. Now look at the two faces $F_i$ and $F_j$:
	\begin{align*}
		F_i: \quad & c(v_0) \xrightarrow{m_1} c(v_1) \xrightarrow{m_2}  \cdots \xrightarrow{m_{d-1}} c(v_{d-1}) \\
		F_j: \quad & c(w_0) \xrightarrow{p_1} c(w_1) \xrightarrow{w_2}  \cdots \xrightarrow{p_{d-1}} c(w_{d-1})
	\end{align*}
	Now by the remark above we have that the orientation of these faces is opposite.
\end{proof}



\begin{proof}
	Assume for the sake of contradiction that we have a cycle in the \EndOfLine\ instance. Let ${\left(S_i\right)}_{i=1}^k$ be a cycle of simplices for colors $C = \set{0, 1, 2}$, and ${\left(F_i\right)}_{i=1}^k$ be the rainbow transition faces. We will show that this leads to a contradiction. \par
	Consider the faces $L_i^{c}$ which are the faces spanned by vertices colored not $c$ in $S_i$. Formally:
	\begin{align*}
		L_i^c & = \set{v \in S_i \mid c(v) \neq c} \quad \text{for $c \in C$}
	\end{align*}
	Notice that these $L_i^c$'s and are always either 2-dimensional or 1-dimensional simplices. Now for the sake of simplicity assume that we remove all 1-dimensional edges from these sequences, and only consider the 2-dimensional faces. Notice that this is in itself a valid oriented simplicial sequence, and in particular a cycle, by \cref{claim:orientability_inside_simplex}.

	This means that we have three cycles of faces. Now for each of these face sequences we look at the transion edges ${\left(Q_i^c\right)}_i$ of ${\left(L_i^c \right)}_i$\marginnote{Recall that this means that:
		\begin{align*}
			Q_i^c = L_i^c \cap L_{i+1}^c
		\end{align*}}.
	We then have that by \cref{lem:consistent_orientation_of_transition_faces} that the orientations of the sequence ${\left(Q_i^c\right)}_i$ is constant.

	Now we argue that on one of the cycles ${\left(L_i^c\right)_i}$ we must have a violation of the constraints we previously enumerate. First notice that because the cycle moves in at least two dimensions there is a $c \in \set{1, 2}$ such that the cycle moves in dimension $c$. This means that there is a hyperplane $H$ obtained by fixing dimension $c$ which is crossed twice by the cycle. Let $F_i$ and $F_j$ be the transition faces which cross the hyperplane $H$.

	Now by \cref{lem:ordering_of_vertices_in_transition_faces} we must have that the vertex colored $0$ must be larger than the vertex colored $c$ in the transition faces $F_i, F_j$. We now proceed by case distinction according to the color of $c$. Recall that by construction $c \in \set{1, 2}$.

	\begin{case}{1}
		We have $c=1$. Then we must have that the vertex colored $0$ is larger than the vertex colored $1$, and the vertex colored $2$ is larger than the vertex colored $1$ in the transition faces $F_i, F_j$, by \cref{lem:ordering_of_vertices_in_transition_faces}. This means that the vertex colored $0$ must appear after $\xrightarrow{3}$ in the transition faces $F_i, F_j$.

		Now we have that the orientation of the transition faces $F_i, F_j$ is opposite by \cref{lem:orientation_of_transition_faces}. This strongly restricts the possible orientations of $F_i, F_j$. The valid orientations of faces are:
		\begin{align*}
			(1): \quad & 1 \xrightarrow{2} 2 \xrightarrow{3} 0 \\
			(2): \quad & 1 \xrightarrow{3} 2 \xrightarrow{2} 0 \\
			(3): \quad & 1 \xrightarrow{3} 0 \xrightarrow{2} 2
		\end{align*}
		Among these orientations the faces of type $(2)$ have negative orientation and the faces of type $(1)$ and $(3)$ have positive orientation. Because they must have opposite orientations, this means that without loss of generality we can assume that $F_j$ is of the form:
		\begin{align*}
			F_j: \quad & 1 \xrightarrow{3} 2 \xrightarrow{2} 0
		\end{align*}
		This means that the face $F_i$ has the form $(1)$ or $(3)$. Notice that in both cases the vertex colored $2$ comes after the edge $\xrightarrow{2}$, whereas in $F_j$ the vertex colored $2$ comes before the edge $\xrightarrow{2}$. This means that we must also move in dimension 2. Now consider the face
	\end{case}
\end{proof}

%%%%%%%%%%%%%%%%%%%%%%%%%%%%%%%%%%%%%%%%%%%%%%%%%%%%%%%%%%%%%%%%%%%%%%%%%%%%%%%%%%%%%%%%%%%%%%%%%
%Discussing the EOPL Reduction
%%%%%%%%%%%%%%%%%%%%%%%%%%%%%%%%%%%%%%%%%%%%%%%%%%%%%%%%%%%%%%%%%%%%%%%%%%%%%%%%%%%%%%%%%%%%%%%%%
\section{Discussing the reduction of \Tarskistar\ to \EndOfPotentialLine}[Discussing the reduction to \EOPL]

We have now shown that there are no cycles in the \EndOfLine\ instance which arise from monotone \Tarskistar\ instances. This does not directly imply a direction onto \EndOfPotentialLine. Inuitively an \EndOfLine\ instance without cycles should be equivalent to an \EndOfPotentialLine\ instance. But in \EndOfPotentialLine\ the non-existence of a cycle is given by a potential which grows monotonically along the lines.

The challenge in our case is to define such a potential. We briefly want to give some inuition for why this is not easy. A first naive attempt, would be to define the potential as something similar to the sum of the coordinates of a vertex of the simplex\marginnote{Recall that this is exactly the same potential as we used to reduce \Tarski\ onto \Localopt.}. This will not work in our case though, because it can very well happen that the path though the simplicial complex is not monotone with respect to the sum of the coordinates, in particular we can cross a hyperplane multiple times, which would not be possible if the sum of the coordinates was a valid potential.

This means that we need to be more creative in our definition of the potential. The idea is to decide on how we orient our potential locally. We start by giving every cell of the simplicial complex coordinates, these will be the coordinates of the smallest vertex of cell. Now look at the colors of the vertices of the cell.

\section{Reducing \Tarskistar\ to \EndOfPotentialLine}[Reducing \Tarskistar\ to \EndOfPotentialLine]

We will now present the reduction from \Tarskistar\ to \EndOfPotentialLine. Before we start let us make a few useful remarks. Recall that the main difference between \EndOfLine\ and \EndOfPotentialLine\ is that in the latter we have a potential which grows monotonically along the lines, which makes the existence of cycles impossible. As we have discussed the \EndOfLine\ instance which arise from \Tarskistar\ instances can have cycles. We will show that finding a cycle also helps us make progress towards finding a solution to the \Tarskistar\ instance.

\begin{lemma}[Cycles almost yield \Tarskistar\ solutions]\label{lem:cycles_almost_yield_solutions}
	Assume that we have a cycle in the \EndOfLine\ instance which arises from a monotone \Tarskistar\ instance where we walked through faces colored $C$, where $C$ is only missing one color $c \in \set{1, \dots, d}$. Then we easily obtain two points $x, y$ such that $x \leq y$ and:
	\begin{itemize}
		\item $x$ is a forward point, i.e.~$c(x) = 0$ or equivalently $x \leq f(x)$;
		\item $y$ is \emph{almost} a backward point in the following sence: $f(y)[i] \leq y[i]$ for all $i \in C \setminus \set{0}$.
	\end{itemize}
\end{lemma}

This lemma means that we almost obtain a solution to the \Tarskistar\ instance. The only thing that is missing is that $y$ is not a backward point in dimension $c$. We will discuss how we can remedy this issue in the following, once we have proven the lemma.

\begin{proof}
	Assume that we have a cycle in the \EndOfLine\ instance which arises from a monotone \Tarskistar\ instance where we walked through faces colored $C$, where $C$ is only missing one color $c \in \set{1, \dots, d}$. This means that we have a valid sequence of simplices ${\left(S_i\right)}_{i=1}^k$ and a valid sequence of faces ${\left(F_i\right)}_{i=1}^k$, such that for two dimensions $l, q$ we have that the cycle moves forward in dimension $l$, and then backwards in dimension $q$. Now this means that we have two orthogonal faces $F_a, F_b$ that are faces of the same cell, and such that $F_a$ is missing dimension $l$ and $F_b$ is missing dimension $q$.

	Set $x$ to be the vertex of $F_a$ which is colored $0$ and set $y$ to be the smallest vertex of the cell. Now by definition $x$ is a forward point. Now we want to show that $y$ is almost a backward point. Let $i$ be any dimension in $C \setminus \set{}$, and consider the following cases:
	\begin{case}{1}
		If $i \neq l$ then we have that that there is a vertex $z$ colored $i$ in $F_a$ such that:
		\begin{align*}
			z[i] - 1 \leq y[i] \leq z[i]
		\end{align*}
		Because $z$ shows strictly backwards in dimension $i$ we have that $f(y)[i] \leq y[i]$. This shows the claim in this case.
	\end{case}
	\begin{case}{2}
		If $i = l$ then we have that there is a vertex $z$ colored $i$ in $F_b$ such that:
		\begin{align*}
			z[i] - 1 \leq y[i] \leq z[i]
		\end{align*}
		Because $z$ shows strictly backwards in dimension $i$ we have that $f(y)[i] \leq y[i]$. This shows the claim in this case.
	\end{case}
	This shows the lemma in all cases.
\end{proof}

Now as we already remarked this is almost a solution to the \Tarskistar\ instance. The only thing that is missing is that $y$ is not a backward point in dimension $c$. We will now show how we can remedy this issue. The key idea is to use the fact that we have some freedom in how we can choose the color set $C$, in particular how to choose the missing color $c$.

Now we are ready for the reduction of \Tarskistar\ to \EndOfPotentialLine. We will construct $d$ \Tarskistar\ instances, which we reduce to \EndOfLine\ instances, but each of them using an individual color set $C$. For each choice of $c \in \set{1, \dots, d}$, we construct one instance where $c$ is the missing color in $C$. We want these to be \EndOfPotentialLine\ instances, which means we need to define a potential. The obvious choice of a potential is to use the sum of the coordinates of the cell. This is not quite sufficient, as we need to make sure that the potential is defined for every \emph{simplex} and not simply for every \emph{cell}. Hence we need to add a component taking into account the simplices themselves. We propose the following potential $\phi$ for a simplex $S$, with permutation $\pi \in S_d$ of the dimensions. Sort all the permutation in lexicographic order, and let $\phi(\pi)$ be the rank of $\pi$ in this order. Now we can define the potential of a simplex $S$ with smallest vertex $x$:
\begin{align*}
	\phi(S) = \sum_{i=0}^{d-1} x[i] + \phi(\pi)
\end{align*}
Now notice that if the potential decreases at any point, this indicates that we were moving forwards in some direction and are now moving backward, putting us in the situation of \cref{lem:cycles_almost_yield_solutions}.

Now using this potential we obtain $d$ \EndOfPotentialLine\ instances. Solving them yields yields either:
\begin{itemize}
	\item A solution to the \Tarskistar\ instance,
	\item $d$ violations of the potential, which can be used to find forward points $x_1, x_2, \dots, x_d$ and almost backward points $y_1, y_2, \dots, y_d$, such that $f(y_i)[j] \leq y_i[j]$ for $j \neq i$.
\end{itemize}
Now notice that we remind the reader of the join and meet operations of points, which we will use shortly.
\begin{definition}[Join and meet of points]
	Let $x, y$ be two points. We define the join of $x$ and $y$ as the point $z$ such that $z[i] = \max(x[i], y[i])$ for all $i$. We define the meet of $x$ and $y$ as the point $z$ such that $z[i] = \min(x[i], y[i])$ for all $i$.
\end{definition}
To conclude simply notice that the join $x$ of the forward points $x_1, x_2, \dots, x_d$ is a forward point, and the meet $y$ of the almost backward points $y_1, y_2, \dots, y_d$ is a backward point. This shows that we can find a solution to the \Tarskistar\ in either $L_{\geq x}$ or $L_{\leq y}$, reducing the search area by at least half. Repeating this at most $\BigO{d\log{N}}$ times yields a solution.

\section{Consequences}

Notably 