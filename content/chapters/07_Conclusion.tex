\chapter{Conclusion}\label{ch:conclusion}

This thesis investigated the \Tarski\ problem and its position within the broader context of total search problems, specifically in the \TFNP\ complexity class. We began by summarizing the established results that position \Tarski\ within \PLS\ and ${\P}^{\PPAD}$. Following this, we introduced a new variant, \Tarskistar, to which the original \Tarski\ problem Turing-reduces. We demonstrated that this new problem reduces to \EndOfLine\ by applying Sperner's Lemma, deviating from the classical approach that relies on Brouwer's fixed-point theorem. This included a complete proof showing that \Sperner\ reduces to \EndOfLine\ thus giving an alternative proof that \Tarski\ belongs to \PPAD\@.

Our goal was to understand why \Tarski\ is situated within \EOPL\ and to construct a reduction from \Tarski\ to the \EndOfPotentialLine\ problem. We discussed the challenges of linking existing results to accomplish this and proposed an alternative approach. By analyzing the structure of \Tarski-instances, particularly regarding the coloring of the lattice, we identified potential pathways for future research aimed at finding a direct reduction from \Tarski\ to \EndOfPotentialLine. We hoped that our reduction to \EndOfLine\ could be turned into a reduction to an \EndOfPotentialLine-instance, by showing that the \EndOfLine-instance does not have cycles. Our plan did not succeed as we found a three-dimensional cycle.

Hence, we did not ultimately achieve a direct reduction but think that working with colored lattices is a promising idea. We also argued why the goal should be to find a reduction from \Tarskistar\ to \EndOfPotentialLine\@, instead of working directly with the original \Tarski\ problem.